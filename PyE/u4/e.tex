\documentclass[12pt,fleqn]{article}
\usepackage{fullpage}
\usepackage{amsmath}
\usepackage{amssymb}
\title{práctica 4}
\author{martín rossi}
\date{}
\begin{document}
\maketitle
\textbf{1.}

recorrido: $R_x=\{2,3,4,5,6,7,8,9,10,11,12\}$

probabilidad puntual:
\[p(X=i)=\frac{1}{36}, i\in\{2,12\}\]
\[p(X=i)=\frac{1}{18}, i\in\{3,11\}\]
\[p(X=i)=\frac{1}{12}, i\in\{4,10\}\]
\[p(X=i)=\frac{1}{9}, i\in\{5,9\}\]
\[p(X=i)=\frac{5}{36}, i\in\{6,8\}\]
\[p(X=7)=\frac{1}{6}\]

probabilidad acumulada:
\[
  F(t)=
  \begin{cases}
    0&t<2\\
    \frac{1}{36}&3 \le t < 4\\
    \frac{1}{12}&4 \le t < 5\\
    \frac{1}{6}&5 \le t < 6\\
    \frac{5}{18}&6 \le t < 7\\
    \frac{5}{12}&7 \le t < 8\\
    \frac{13}{18}&8 \le t < 9\\
    \frac{5}{6}&9 \le t < 10\\
    \frac{11}{12}&10 \le t < 11\\
    \frac{35}{36}&11 \le t < 12\\
    1&12 \le t
  \end{cases}
\]
\[E(X)=\sum_{i=2}^{12} i*p(X=i)=7\]
\[Var(X)=E(X^2)-E(X)^2=\frac{329}{6}-49=\frac{35}{6}\]
\[SD(X)=\sqrt{Var(X)}=2.4152\]

\textbf{2.}

\textbf{a.}\[p(T=3)=\frac{1}{3}, p(T=4)=\frac{1}{6}, p(T=5)=\frac{1}{6}, p(T=6)=\frac{1}{3}\]

\textbf{b.}\[P(3<T \le 5)=P(T=4)+P(T=5)=\frac{1}{6}+\frac{1}{6}=\frac{1}{3}\]

\textbf{c.}\[E(T)=\sum_{i=3}^{6} i*p(T=i)=3*\frac{1}{3}+4*\frac{1}{6}+5*\frac{1}{6}+6*\frac{1}{3}=\frac{9}{2}\]
\[Var(T)=E(T^2)-E(T)^2=\frac{131}{6}-\frac{81}{4}=\frac{19}{12}\]
\[SD(T)=\sqrt{Var(T)}=1.2583\]

\textbf{3.}

\textbf{a.}\[P(X=1)=\frac{1}{8},P(X=2)=\frac{1}{4},P(X=3)=\frac{3}{8},P(X=4)=\frac{1}{4}\]

\textbf{b.}\[P(1 \le X \le 3)=\frac{3}{4},P(X<3)=\frac{3}{8},P(X>1.4)=\frac{7}{8}\]

\textbf{4.}

\textbf{a.}\[Y \sim Bin(3,0.05)\]

\textbf{b.}
\begin{align*}
  P(Y>1)&=P(Y=2)+P(Y=3)\\
        &=\binom{3}{2}*0.05^2*0.95+\binom{3}{3}*0.05^3\\
        &=0.0071+0.0001\\
        &=0.0072
\end{align*}

\textbf{5.}

\textbf{a.}
\[R_z=\mathbb{N}\]

\textbf{b.}
\[P(Z=5)=q^4*p=0.95^4*0.05=0.0407\]

\textbf{6.}

con reposicion:
\[X \sim Bin(3,\frac{1}{5})\\
  F(x)=P(X \le x)=\begin{cases}
    0&x<0\\
    0.512&x=0\\
    0.896&x=1\\
    0.992&x=2\\
    1&x \ge 3
  \end{cases}
\]

sin reposicion:
\[P(X=k)=\frac{\binom{4}{k}\binom{16}{3-k}}{1140}, k=0,1,2,3\\
  F(x)=P(X \le x)=\begin{cases}
    0&x<0\\
    0.4912&x=0\\
    0.9123&x=1\\
    0.9965&x=2\\
    1&x \ge 3
  \end{cases}
\]

\textbf{7.}
\[X: \textrm{cantidad de piezas defectuosas}\]
\[p=\frac{3}{25}\]
\[P(X=1)=\frac{\binom{3}{1}\binom{22}{4}}{\binom{25}{5}}=0.4130\]
\[100*P(X=1)=100*0.4130=41.3\]

\textbf{8.}

\textbf{a.}
\[P(X=k)=(1-p)^{(k-1)}*p\]

\textbf{b.}
\[P(X=5)=(1-p)^4*p\]

\textbf{c.}
\[\frac{\delta}{\delta p}[(1-p)^4*p]=-(5*p-1)(1-p)^3\]
\begin{align*}
  -(5*p-1)(1-p)^3&=0\implies\\
  5*p-1&=0\implies\\
  p&=0.2
\end{align*}

en $p=0.2$ alcanza el maximo
\end{document}