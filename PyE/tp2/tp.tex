\documentclass[12pt]{article}
\usepackage{fullpage}
\usepackage{amsmath}
\usepackage{multirow}
\title{\huge{\textbf{Ejercicios unidad 3}}}
\author{Martín Rossi}
\date{}
\begin{document}
\maketitle
\textbf{1)}
Se definen los sucesos:

$T_0=\textrm{``se emite un 0''}$

$T_1=\textrm{``se emite un 1''}$

$R_0=\textrm{``se recibe un 0''}$

$R_1=\textrm{``se recibe un 1''}$

El espacio muestral sería:

$S=\{(T_0,R_0),(T_0,R_1),(T_1,R_0),(T_1,R_1)\}$

y el suceso $E=\textrm{``se comete un error en la transmisión"}$:

$E=\{(T_0,R_1),(T_1,R_0)\}$

entonces $P(E)=\frac{\#E}{\#S}=\frac{2}{4}=\frac{1}{2}$

\textbf{2)}
Se definen los sucesos:

$I_1: \textrm{"error importante en la primer prueba"}$

$M_1: \textrm{"error menor en la primer prueba"}$

$N_1: \textrm{"ningún error en la primer prueba"}$

$I_2: \textrm{"error importante en la segunda prueba"}$

$M_2: \textrm{"error menor en la segunda prueba"}$

$N_2: \textrm{"ningún error en la segunda prueba"}$

$P(I_1)=0.6$

$P(M_1)=0.3$

$P(N_1)=0.1$

\textbf{a)}
Usando estas probabilidades junto con la formula de probabilidad condicional

$P(A|B)=P(A \cap B)/P(B)$ se forma la tabla de intersecciones:\\

\begin{tabular}{r r r r r}
  &&\multicolumn{3}{l}{\textbf{Tipo de error segunda prueba}}\\
  &&Importante&Menor&Ninguno\\
  \textbf{Tipo de error}&Importante&0.09&0.15&0.06\\
  \textbf{primera prueba}&Menor&0.06&0.18&0.36\\
  &Ninguno&0&0.02&0.08\\
\end{tabular}\\

\textbf{b)}
Se puede condicionar la probabilidad por el resultado de la primer prueba.

Como $S=I_1 \cup M_1 \cup N_1$, se calcula $P(I_2)$ con los valores de la tabla:
\begin{align*}
  P(I_2)&=P(I_2|I_1)P(I_1)+P(I_2|M_1)P(M_1)+P(I_2|N_1)P(N_1)\\
        &=0.3*0.6+0.1*0.3+0*0.1\\
        &=0.18+0.03+0\\
        &=0.21
\end{align*}

\textbf{c)}
\begin{align*}
  P(M_1|I_2)&=P(M_1 \cap I_2)/P(I_2)\\
            &=0.06/0.21\\
            &=0.28
\end{align*}

\textbf{d)}
Los resultados de la primer prueba no son independientes a los de la segunda. Una vez hechas las dos pruebas los resultados de la segunda condicionan los de la primera.

\textbf{3)}
Se definen los sucesos:

$E: \textrm{``una persona tiene la enfermedad''}$

$A: \textrm{``el test da positivo''}$

tenemos los siguientes datos:

$P(A|E)=0.9$

$P(A|\overline{E})=0.05$

$P(E)=0.12$

se calcula $P(E|A)$:

\begin{align*}
  P(E|A)&=\frac{P(A|E)P(E)}{P(A)}\\
        &=\frac{P(A|E)P(E)}{P(A|E)P(E)+P(A|\overline{E})P(\overline{E})}\\
        &=\frac{0.9*0.12}{0.9*0.12+0.05*0.88}\\
        &=0.71
\end{align*}
\end{document}