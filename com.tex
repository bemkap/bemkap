% Created 2021-09-20 lun 23:40
% Intended LaTeX compiler: pdflatex
\documentclass[11pt]{article}
\usepackage[utf8]{inputenc}
\usepackage[T1]{fontenc}
\usepackage{graphicx}
\usepackage{grffile}
\usepackage{longtable}
\usepackage{wrapfig}
\usepackage{rotating}
\usepackage[normalem]{ulem}
\usepackage{amsmath}
\usepackage{textcomp}
\usepackage{amssymb}
\usepackage{capt-of}
\usepackage{hyperref}
\usepackage{fullpage}
\date{\today}
\title{}
\hypersetup{
 pdfauthor={},
 pdftitle={},
 pdfkeywords={},
 pdfsubject={},
 pdfcreator={Emacs 27.2 (Org mode 9.4.4)}, 
 pdflang={English}}
\begin{document}

\tableofcontents

\section{introduccion}
\label{sec:orgcfcc485}
\subsection{hardware}
\label{sec:org4b83ee0}
\subsection{sofware de red}
\label{sec:org6ac81c9}
\subsubsection{jerarquia de protocolos}
\label{sec:orgfe5638b}
\begin{itemize}
\item organizacion por capas. cada capa tiene una funcion diferenciada e independiente
\item intercambio de mensajes segun el protocolo de cada capa
\item en realidad los mensajes bajan hasta la capa inferior (medio fisico), donde se realiza la comunicacion
\item interfaz bien definida para comunicacion entre capas
\item arquitectura de red: conjunto de capas y protocolos
\item pila de protocolos: lista de protocolos usados por una arquitectura
\end{itemize}
\subsubsection{aspectos de diseño para cada capa}
\label{sec:org18cc93a}
\begin{itemize}
\item codigos de deteccion (y posible correccion) de errores
\item enrutamiento: eleccion de una ruta para enviar informacion
\item distribucion de protocolos en capas
\item mecanismos para embalar, desembalar y transmitir
\item escalabilidad
\item asignacion eficiente de recursos
\item uso del ancho de banda (multiplexado estadistico, fraccion fija)
\item control de flujo
\item confidencialidad, autenticacion e integridad
\end{itemize}
\subsubsection{tipos de servicios}
\label{sec:orgb35e689}
\begin{enumerate}
\item orientados a la conexion
\label{sec:org2ba7b7c}
\begin{itemize}
\item se establece la conexion, se usa y se libera
\item en la mayoria de los casos se preserva el orden
\item como una linea telefonica
\end{itemize}
\item no orientados a la conexion
\label{sec:org16d505d}
\begin{itemize}
\item cada mensaje lleva la direccion de destino completa
\item cada mensaje es enrutado en forma independiente
\item como el sistema postal
\end{itemize}
\item confiables
\label{sec:org78d48cd}
\begin{itemize}
\item nunca pierden datos
\item acuse de recibo
\item introduccion de sobrecarga y retardos
\end{itemize}
\item no confiables
\label{sec:orgc694855}
\begin{center}
\begin{tabular}{lll}
 & confiable & no confiable\\
conexion & secuencia de mensajes & voz sobre ip\\
 & flujo de bytes & \\
no conexion & mensajes de texto & mails\\
\end{tabular}
\end{center}
\end{enumerate}
\subsubsection{relacion entre servicios y protocolos}
\label{sec:orgc18edd1}
\begin{itemize}
\item un servicio se define como un conjunto de primitivas que una capa proporciona a la que esta encima de ella
\item el servicio define el que pero no el como
\item protocolo son las reglas de formato y significado de los paquetes o mensajes que se intercambian en la misma capa
\item servicio se relaciona con las interfaces entre capas
\item protocolo se relaciona con los paquetes que se envian entre distintas maquinas
\end{itemize}
\subsection{modelos de referencia}
\label{sec:org4632a9a}
\subsubsection{modelo osi}
\label{sec:orgf4c69d9}
\begin{enumerate}
\item capa fisica
\label{sec:org3b8470b}
\begin{itemize}
\item transmision de bits puros a traves de un canal de transmision
\item busca que lleguen los mismos bits que salieron
\item señales electricas para representar un bit
\item como se establece y se termina una comunicacion
\end{itemize}
\item capa de enlace de datos
\label{sec:orgd5e8763}
\begin{itemize}
\item transforma los bits puros en una linea que este libre de errores para la capa de red
\item divide los datos en tramos
\item control de transmision para emisores rapidos y receptores lentos
\end{itemize}
\item capa de red
\label{sec:orga9bcddc}
\begin{itemize}
\item como se encaminan los paquetes del origen al destino
\item las rutas se basan en tablas estaticas o dinamicas
\item manejo de congestion
\item solucionar problemas para conectar reder heterogeneas
\end{itemize}
\item capa de transporte
\label{sec:orgbd62a72}
\begin{itemize}
\item aceptar datos de la capa superior, dividirlos en unidades mas pequeñas, pasar los datos a la capa de red y asegurar que las piezas lleguen al otro extremo
\item es una verdadera capa de extremo a extremo, a diferencia de las mas bajas
\end{itemize}
\item capa de sesion
\label{sec:org77f34b5}
\begin{itemize}
\item control de dialogo
\item manejo de tokens
\item sincronizacion
\end{itemize}
\item capa de presentacion
\label{sec:orgdf509eb}
\begin{itemize}
\item se enfoca en la sintaxis y la semantica de la informacion transmitida
\item maneja estructuras abstractas para intercambiar datos entre computadoras con diferentes representaciones de datos
\end{itemize}
\item capa de aplicacion
\label{sec:orgb67022e}
\begin{itemize}
\item protocolos que los usuarios necesitan
\end{itemize}
\end{enumerate}
\subsubsection{modelo tcp/ip}
\label{sec:orgcf05cda}
\begin{enumerate}
\item capa de enlace
\label{sec:org5c05ed1}
\begin{itemize}
\item capa sin conexion que opera a traves de distintas redes
\item describe que enlaces se deben llevar a cabo para cumplir con las necesidades de esta capa
\end{itemize}
\item capa de interred
\label{sec:org52aebbc}
\begin{itemize}
\item permite que los host inyecten paquetes en cualquier red y que viajen independientemente a su destino
\item analogo al sistema de correo
\item define un formato de paquete y un protocolo oficial llamado ip y uno complementario llamado icmp
\item el ruteo de paquetes es el principal aspecto, y la congestion
\end{itemize}
\item capa de transporte
\label{sec:org4972ec9}
\begin{itemize}
\item permite que entidades en la misma capa mantengan una conversacion
\item tcp, udp
\end{itemize}
\end{enumerate}
\subsubsection{capa de aplicacion}
\label{sec:org586d88b}
\begin{itemize}
\item reemplaza las capas de presentacion, sesion y aplicacion del modelo osi
\item telnet, ftp, smtp, dns, http
\end{itemize}
\subsubsection{comparacion tcp/ip osi}
\label{sec:orgdddfaeb}
\begin{itemize}
\item osi fue inventado antes que los protocolos, por eso es mas general. pero los diseñadores no sabian que funcionalidades colocar en cada capa
\item con tcp/ip paso al reves. los protocolos encajaron perfectamente, pero no era util para describir redes que no fueran tcp/ip
\item osi tiene 7 capas, tcp/ip tiene 4
\end{itemize}
\subsubsection{defectos de osi}
\label{sec:org5a33488}
\begin{itemize}
\item mala sicronizacion: para cuando se desarrollaron los protocolos osi, tcp/ip ya se estaba usando lo suficiente como para que los distribuidores no quisieran apoyar otra pila
\item mala tecnologia: el modelo es muy complejo. las capas de sesion y presentacion estan casi vacias, las de red y enlace llenas. son dificiles de implementar e ineficientes.
\item malas implementaciones: por su complejidad las primeras implementaciones eran lentas y pesadas. despues mejoraron pero la imagen quedo
\item malas politicas: osi se asocio con el gobierno estadounidense y tcp/ip con unix
\end{itemize}
\subsubsection{defectos de tcp/ip}
\label{sec:org0da2a29}
\begin{itemize}
\item no se diferencian bien los conceptos de servicio, interfaz y protocolo
\item el modelo no es para nada general
\item la capa de enlace no es una capa sino una interfaz
\item no distingue la capa de enlace y la fisica
\end{itemize}
\section{capa fisica}
\label{sec:org80d27c0}
\subsection{conceptos}
\label{sec:org444ca46}
\begin{itemize}
\item serie de fourier
\item ancho de banda
\item banda base, pasa-banda
\item teorema de nyquist, teorema de shannon
\item relacion señal ruido S/N
\end{itemize}
\subsection{medios de transmision guiados}
\label{sec:orgf496a9c}
\subsubsection{medios magneticos}
\label{sec:org137717c}
\begin{itemize}
\item guardar la informacion en una cinta o medio removible y mandarlo fisicamente
\item \emph{nunca subestime el ancho de banda de una camioneta repleta de cintas que viaje a toda velocidad por la carretera}
\end{itemize}
\subsubsection{par trenzado}
\label{sec:org53087a0}
\begin{itemize}
\item dos cables de cobre aislados
\item trenzados porque en paralelo forman una antena
\item la señal se transmite como la diferencia de voltaje entre los dos cables
\item el ruido afecta a los dos cables por igual, el diferencial se mantiene
\item sistema telefonico
\item informacion analogica o digital
\item el ancho de banda depende del grosor de los cables y la distancia. hasta varios mbps
\item ethernet usa cuatro, uno para cada direccion
\item hasta cat 6: utp (unshielded twisted pair). cat 7: stp
\end{itemize}
\subsubsection{cable coaxial}
\label{sec:orge0a2195}
\begin{itemize}
\item mejor blindaje y mayor ancho de banda que los tp, pero mas caro
\end{itemize}
\subsubsection{lineas electricas}
\label{sec:orgaa2a0de}
\begin{itemize}
\item las compañias las han utilizado para comunicacion de baja velocidad
\item uso en el hogar para controlar dispositivos
\item dificil porque el cableado de las casas no esta hecho para enviar señales a alta frecuencia
\end{itemize}
\subsubsection{fibra optica}
\label{sec:org80ed464}
\begin{itemize}
\item lan, internet y ftth
\item un pulso de luz indica 1, la ausencia 0
\item cuando la luz pasa de un medio a otro (silice a aire) se refracta. el grado depende de los indices de refraccion de los medios. y para cualquier angulo mayor a un angulo critico la luz rebota completamente en el silice
\item fibra multimodal: varios rayos de luz en una fibra
\item fibra monomodo: un solo rayo de luz por fibra que es mucho mas angosta
\item tres bandas: 0.85 1.3 y 1.55 micras. anchos de banda de 25000 a 30000 ghz. la primera tiene mas atenuacion
\item fuentes: led y laser
\end{itemize}
\subsection{transmision inalambrica}
\label{sec:orga8e11df}
\subsubsection{espectro electromagnetico}
\label{sec:org793362d}
\begin{itemize}
\item los electrones se mueven y crean ondas electromagneticas
\item las ondas viajan siempre a la velocidad de la luz
\item \(\lambda f=c\)
\item espectro directo con salto de frecuencia: transmision dificil de detectar y bloquear. militares, bluetooth, versiones anteriores de 802.11
\item espectro directo de secuencia directa: multiples señales comparten ancho de banda. cdma, gps, 802.11b
\item uwb
\end{itemize}
\subsubsection{radiotransmision}
\label{sec:org5766621}
\begin{itemize}
\item las ondas de radio son faciles de generar, recorren largas distancias y penetran edificios
\item son omnidireccionales
\item las propiedades dependen de la frecuencia. baja frecuencia: cruzan obstaculos pero se reduce la potencia rapidamente. alta frecuencia: viajan en linea recta y rebotan en obstaculos
\item ondas de alta frecuencia son absorbidas por la lluvia y otros obstaculos
\item como recorren grandes distancia la interferencia es un problema
\item estan reguladas por los gobiernos
\item vlf, lf y mf siguen la curvatura de la tierra. hf van en linea recta y rebotan en la ionosfera, tambien son absorbidas por la tierra
\end{itemize}
\subsubsection{transmision por microondas}
\label{sec:orgdbce5e8}
\begin{itemize}
\item relacion S/N alta, pero las antenas deben estar alineadas
\item microondas no atraviesan bien los edificios
\item comunicacion telefonica, celulares, television. lo que provoco escasez de espectro
\end{itemize}
\subsubsection{transmision infrarroja}
\label{sec:org40360f9}
\begin{itemize}
\item comunicacion de corto alcance
\item no atraviesan objetos
\end{itemize}
\subsubsection{tranmision por ondas de luz}
\label{sec:org1fcdfe2}
\begin{itemize}
\item señalizacion optica mediante laser
\item gran ancho de banda a bajo costo y seguro. pero muy dificil de apuntar
\end{itemize}
\subsection{satelites de comunicacion}
\label{sec:orgec328f4}
\begin{itemize}
\item un satelite es un enorme repetidor de microondas con varios transpondedores. transmite en modo \textbf{tublo doblado}
\item posicion de los satelites limitadas por el cinturon de van allen
\end{itemize}
\subsubsection{satelites geoestacionarios}
\label{sec:org9db387c}
\begin{itemize}
\item satelites que orbitan a la misma velocidad de la que rota la tierra. parecen inmoviles desde el suelo
\item los primeros tenian un solo haz de luz que iluminaba la tierra, lo que se conoce como huella
\item actualmente tienen multiples haces que se enfocan en una pequeña area geografica. estos son los haces puntuales
\item vsat: terminales muy pequeñas que se utilizan para la transmision de tv
\item los vsat no se pueden comunicar entre ellos por su baja potencia. para ello usan de intermediario potentes estaciones en la tierra
\item aunque las señales viajen a la velocidad de la luz, dada las distancias tienen mas retardo que las comunicaciones terrestres
\item los satelites son medios de difusion por naturaleza
\end{itemize}
\subsubsection{ventajas de los satelites sobre la fibra optica}
\label{sec:org52a18d5}
\begin{itemize}
\item cuando se requiere un despliegue rapido, ganan los satelites
\item los satelites pueden enviar a cualquier parte del mundo
\item un mensaje que envia un satelite lo pueden recibir miles de estaciones al mismo tiempo
\end{itemize}
\subsection{modulacion digital y multiplexacion}
\label{sec:org846fb53}
\begin{itemize}
\item modulacion digital: proceso de convertir bits en la señal que los representan
\item transmision en banda base: la señal ocupa una frecuencia desde 0 hasta un valor maximo que depende de la tasa de señalizacion. comun en cables
\item transmision pasa-banda: la señal ocupa una banda de frecuencias alrededor de la frecuencia de la señal portadora. comun en inalambrico y optico
\item multiplexacion: a compartir varias señales por un mismo canal
\end{itemize}
\subsubsection{transmision en banda base}
\label{sec:orgc02bb38}
\begin{itemize}
\item NRZ(non-return-to-zero): voltaje positivo para el 1 y uno nulo para el 0
\item el receptor muestrea a intervalos regulares y convierte de nuevo a bits. la señal no se vera igual a la que se envio por el ruido y el canal
\item eficiencia del ancho de banda
\begin{itemize}
\item con nrz la señal puede alternar entre positivo y negativo hasta cada 2 bits. necesita un ancho de banda B/2hz pasa tasa de B bps
\item una estategia es usar mas de 2 niveles de señalizacion. por ejemplo 4 voltajes para representar 2 bits a la vez como un simbolo
\item tasa de bits=tasa de simbolo*bits por simbolo
\item requiere una potencia mayor en el receptor para diferenciar los niveles
\end{itemize}
\item recuperacion del reloj
\begin{itemize}
\item el receptor debe saber cuando termina un simbolo y empieza otro
\item existe un limite en la precision de un reloj para muestrear señales
\item se podria enviar una señal del reloj por otra linea separada, pero seria mejor que si hubiera otra linea se usara para enviar datos
\item un truco seria usar xor entre las dos lineas para enviarlas en una sola. esta es la codificacion manchester y se usaba en ethernet clasico. lo malo es que requiere el doble de ancho de banda
\item una estrategia distinta es codificar los datos para que haya suficientes transiciones en la señal. ya que los problemas suceden en largas suceciones de 0 o 1
\item nrzi: 1 como una transicion y 0 como no hay transicion. usb usa este metodo. largas sucesiones de 1 no tienen problemas, pero de 0 si
\item 4b/5b: se asocian grupos de 4 bits a 5 bits segun una tabla fija, de manera que nunca haya tres 0 seguidos. agrega 25\% de sobrecarga. sobran 16 numeros de 5 bits, algunos se usan para control
\item para asegurar transiciones se puede hacer xor con una secuencia pseudoaleatoria. el receptor decodifica con la misma secuencia. esta debe ser facil de generar
\item pero la aleatorizacion no garantiza transiciones
\end{itemize}
\item señales balanceadas
\begin{itemize}
\item señales que tienen misma cantidad de voltajes positivos como negativos
\item ayuda a proveer transiciones para la recuperacion del reloj
\item codificacion bipolar: se alterna +1 y -1 voltios para el 1 y 0 voltios para el 0. en redes telefonicas ami
\item 8b/10b tambien para codigo balanceado
\end{itemize}
\end{itemize}
\subsubsection{transmision pasa-banda}
\label{sec:org431af33}
\begin{itemize}
\item en canales inalambricos no es practico usar rango de frecuencias que empiecen en 0
\item se puede tomar una señal en banda base que ocupe de 0 a b hz y desplazarla a otra pasa-banda que ocupe de s a s+b hz
\item se puede modular la amplitud (ask), la frecuencia (fsk) o la fase (psk)
\item psk puede ser bpsk (binaria) o qpsk (cuadratura)
\item se pueden combinar y usar mas niveles, comunmente amplitud y fase
\item diagrama de constelacion: forma de visualizar la modulacion combinada ask y psk. qpsk, qam-16, qam-64
\item simbolos adyacentes no deben diferir en muchos bits, porque serian mas suceptibles al ruido. para eso se usa codigo gray
\end{itemize}
\subsubsection{multiplexacion por division de frecuencia}
\label{sec:orge4605cc}
\begin{itemize}
\item fdm: divide el espectro en bandas. cada usuario tiene posesion exclusiva de la banda
\item banda de guarda: exceso de banda que mantiene a los canales separados
\item ofdm: el ancho de banda del canal se divide en muchas subportadoras que envian de manera independiente. cada subportadora esta diseñada para ser 0 en el centro de las adyacentes. 802.11
\end{itemize}
\subsubsection{multiplexacion por division de tiempo}
\label{sec:orgb0664b4}
\begin{itemize}
\item tdm: los usuarios toman turnos y usan todo el ancho de banda, se toman los datos y se agregan al flujo agregado
\item para que funcione debe haber sincronizacion. se puede agregar tiempo de guarda
\end{itemize}
\subsubsection{multiplexacion por division de codigo}
\label{sec:org359769a}
\begin{itemize}
\item cdm: forma de comunicacion de espectro diverso. una señal de banda estrecha se dispersa en una mas amplia. cdma
\item hace la señal mas tolerante a interferencias y permite que señales compartan la misma banda de frecuencia
\item cdma es extraer la señal deseada mientras lo demas se rechaza como ruido
\item cada tiempo de bit de divide en m intervalos llamados chips. en general 64 o 128 chips cada bit. a cada estacion se le asigna una secuencia de chip, un codigo de m bits. para transmitir un 1 envia la secuencia de chip, para el 0 la negacion
\item todas las secuencias de chip son ortogonales por pares
\item si varias estaciones envian al mismo tiempo se suman
\end{itemize}
\section{capa de enlace}
\label{sec:orgff26d6a}
\subsection{cuestiones de diseño}
\label{sec:org9b1ce65}
\begin{itemize}
\item funciones: dar a la capa de red una interfaz de servicios bien definida. manejar errores. controlar flujo
\item toma los datos que obtiene de la capa de red y los encapsula en tramas
\end{itemize}
\subsubsection{servicios dados a la capa de red}
\label{sec:org3278209}
\begin{itemize}
\item transferir datos de la maquina de origen a la de destino
\item 3 servicios razonables
\begin{itemize}
\item sin conexion ni confirmacion de recepcion: tasa de error baja. trafico en tiempo real. ethernet
\item sin conexion con confirmacion: canales no confiables. 802.11 (wifi)
\item con conexion y confirmacion: cada trama esta enumerada. se garantiza que lleguen solo una vez y en orden. canales largos y no confiables. satelites y red telefonia larga
\end{itemize}
\end{itemize}
\subsubsection{entramado}
\label{sec:org2d02fd3}
\begin{itemize}
\item la capa fisica no garantiza que el flujo de bits este libre de errores
\item un metodo es dividir el flujo en tramas discretas y agregarles una suma de verificacion
\item division de tramas
\begin{itemize}
\item conteo de bytes: agrega en el encabezado la cantidad de bytes en la trama. si se altera este valor se pierde la sincronia. rara vez se usa solo
\item bytes bandera con relleno de bytes: cada trama inicia y termina con bytes especiales. si aparece la bandera en los datos se antecede un escape. y si aparece un escape se pone otro escape adelante. simplificacion de ppp
\item bits bandera con relleno de bits: igual a bytes pero sin la restriccion de 1 byte=8 bits. hdlc. usb. se usan 6 bits en 1 para delimitar. cada vez que se ven 5 bits en 1 se agrega un 0
\item violaciones de codificacion de la capa fisica: si se usa por ejemplo 4b/5b en la capa fisica se pueden usar los codigos no utilizados para el inicio y fin de trama
\end{itemize}
\end{itemize}
\subsubsection{control de errores}
\label{sec:orgd19991d}
\begin{itemize}
\item asegurar la entrega de datos confiable: retroalimentacion al emisor de lo que esta ocurriendo del otro lado. positiva y negativa
\item puede desaparecer la trama por completo, o la de retroalimentacion. para eso tambien se usan temporizadores para enviar nuevamente
\item ahora puede que se reciba la misma trama dos veces. para eso se usan numeros de secuencia
\end{itemize}
\subsubsection{control de flujo}
\label{sec:org0c20393}
\begin{itemize}
\item que hacer cuando un emisor envia mas tramas de las que el receptor puede aceptar. ejemplo telefono y sitio web
\item control de flujo basado en retroalimentacion: el receptor envia cuando puede aceptar mas datos
\item control de flujo basado en tasa: el protocolo tiene un mecanismo integrado que limita la tasa de envio
\end{itemize}
\subsection{deteccion y correccion de errores}
\label{sec:org5baef3d}
\begin{itemize}
\item estategia: incluir redundancia en los datos.
\item codigo de correccion de errores: para que el receptor pueda deducir que datos se quisieron enviar. fec
\item codigo de deteccion de errores: para que sepa que hubo un error pero nada mas y solicite retransmision
\item en fibra optica conviene la deteccion porque es rapido reenviar. en canales inalambricos es mejor correccion
\item los bits de redundancia tambien pueden llegar mal. asi que nunca se podran manejar todos los errores
\item los errores en rafaga tienen sus ventajas y desventajas
\end{itemize}
\subsection{protocolos de enlace de datos}
\label{sec:org592bc38}
\subsubsection{paquetes sobre sonet}
\label{sec:orga7b1dfe}
\begin{itemize}
\item sonet se utiliza sobre canales de fibra optica de area amplia
\item ppp se usa para diferenciar paquetes ocasionales del flujo continuo en el que se transportan
\end{itemize}
\subsubsection{ppp}
\label{sec:org6949249}
\begin{itemize}
\item ppp orientado a bytes, hdlc a bits
\item metodo de entramado sin ambiguedades, tambien maneja deteccion de errores
\item protocolo para activar lineas, probarlas, negociar y desactivarlas. lcp
\item mecanismo para negociar opciones de capa de red independientemente del protocolo de red usado
\item uso de banderas como delimitacion y bytes de escape
\item la carga util se mezcla aleatoriamente antes de insertarla en sonet para garantizar mas transiciones que necesita sonet
\item configuracion enlace ppp
\begin{itemize}
\item muerto
\item establecer (cuando hay conexion en la capa fisica): intercambio de paquetes lcp
\item autentificar (si lo anterior fue exitoso): se verifican identidades
\item red: paquetes ncp para configurar la capa de red
\item abrir: intercambio de datos
\item terminar
\end{itemize}
\end{itemize}
\section{subcapa control acceso al medio}
\label{sec:org6f6bb49}
\begin{itemize}
\item los enlaces de red pueden ser punto a punto o difusion
\item subcapa mac es la parte inferior de la de enlace de datos
\end{itemize}
\subsection{problema de asignacion de canal}
\label{sec:org789c291}
\begin{itemize}
\item asignar un solo canal de difusion entre varios usuarios competidores
\end{itemize}
\subsubsection{asignacion estatica}
\label{sec:orgbc2ae48}
\begin{itemize}
\item dividir la capacidad mediante el uso de multiplexacion. cuando hay una pequeña cantidad de usuarios constantes
\item si varia el numero de emisores y ese numero es grande se vuelve ineficiente
\item lo mismo sucede con otras formas estaticas de dividir un canal
\end{itemize}
\subsubsection{supuestos para la asignacion dinamica}
\label{sec:org8dfa9cd}
\begin{itemize}
\item trafico independiente: las estaciones son independientes
\item canal unico: hay un solo canal para todas las comunicaciones
\item colisiones observables: todas las estaciones pueden detectar colisiones. que seran enviadas luego
\item tiempo continuo o ranurado: se puede considerar de las dos maneras
\item deteccion de portadora o sin deteccion: si hay deteccion las estaciones pueden saber si el canal esta en uso. sino mandan y despues determinan si tuvo exito
\end{itemize}
\subsection{protocolos de acceso multiple}
\label{sec:org4904b1b}
\subsubsection{aloha}
\label{sec:org9c0e35c}
\begin{itemize}
\item aloha puro
\begin{itemize}
\item despues de enviar su trama a la computadora central, esta difunde la trama a todas las estaciones. asi el emisor sabe si llego su trama
\item si la trama fue destruida espera un tiempo aleatorio y manda de nuevo
\item cada vez que dos tramas intenten ocupar el canal al mismo tiempo habra colision, por mas que sea un solapamiento pequeño
\end{itemize}
\item aloha ranurado
\begin{itemize}
\item como el metodo puro pero el tiempo se divide en ranuras discretas
\item sincronizacion por medio de una estacion que emita una señal al comienzo de cada intervalo
\end{itemize}
\end{itemize}
\subsubsection{protocolos de acceso multiple con deteccion de portadora}
\label{sec:orge59fe04}
\begin{itemize}
\item csma persistente-1
\begin{itemize}
\item la estacion escucha el canal para ver si alguien esta enviando, sino envia. si ocurre una colision espera y manda de nuevo
\item el retardo de propagacion tiene un efecto importante en las colisiones. esta posibilidad depende del numero de tramas que quepan, o producto de ancho de banda-retardo
\item en lan como el retardo es pequeño, no habra muchas colisiones
\end{itemize}
\item csma no persistente
\begin{itemize}
\item a diferencia del persistente-1 si el canal esta en uso espera un tiempo y repite el proceso. no se queda escuchando constantemente
\item mejor uso del canal pero mayor retardo
\end{itemize}
\item csma persistente-p
\begin{itemize}
\item para canales ranurados
\item si el canal esta inactivo, envia con probabilidad p y espera a la siguiente ranura con probabilidad 1-p
\end{itemize}
\item csma con deteccion de colisiones (csma/cd)
\begin{itemize}
\item base de la clasica ethernet
\item el hardware escucha a la vez que envia. si la señal que recibe es distinta a la que envia, esta ocurriendo una colision
\item periodos alternantes de contencion y transmision con periodos de inactividad que ocurriran cuando todas las estaciones esten en reposo
\end{itemize}
\end{itemize}
\subsection{protocolos libres de colisiones}
\label{sec:org2119d7c}
\subsubsection{protocolo de mapa de bits}
\label{sec:orgc84ca54}
\begin{itemize}
\item cada periodo de contencion consiste en n ranuras
\item las estaciones envian 1 si tienen tramas para enviar en ese periodo pero solo en su ranura correspondiente
\item luego cuando ya hay conocimiento de quien va a mandar mandan las tramas en orden
\item protocolos de revervacion
\end{itemize}
\subsubsection{paso de token}
\label{sec:org587eaae}
\begin{itemize}
\item pasa un pequeño mensaje llamado token de una estacion a otra en un orden determinado. token ring
\item solo puede enviar la que tenga el token
\item cuando la estacion que envio recibe su misma trama la elimina para terminar el ciclo
\item no hace falta que sea un anillo. token bus
\end{itemize}
\subsubsection{conteo descendente binario}
\label{sec:orge117781}
\begin{itemize}
\item anteriores no escalan a redes con miles de estaciones
\item las estaciones que quieren usar el canal envian su direccion binaria y hacen or de todo lo que reciben
\item tan pronto como una estacion ve que una posicion de bit de orden alto, cuya direccion es 0, ha sido sobreescrita por un 1, se da por vencida
\end{itemize}
\subsection{protocolos de contencion limitada}
\label{sec:orgbd65272}
\begin{itemize}
\item en condicion de carga ligera es preferible contencion
\item al reves para libres de colision
\item protocolos de contencion limitada combinan los dos anteriores
\end{itemize}
\subsubsection{protocolo de recorrido de arbol adaptable}
\label{sec:orgc9ca858}
\begin{itemize}
\item en la ranura 0 todas las estaciones intentan adquirir el canal. si una lo logra bien y sino se dividen en dos grupos y se va formando un arbol de decision
\item a mayor carga la busqueda debe iniciar mas abajo en el arbol
\end{itemize}
\subsection{protocolos de lan inalambrica}
\label{sec:org8385e23}
\end{document}