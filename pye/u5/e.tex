\documentclass[12pt,fleqn]{article}
\usepackage{fullpage}
\usepackage{amsmath}
\usepackage{amssymb}
\title{práctica 5}
\author{martín rossi}
\date{}
\begin{document}
\maketitle

\textbf{1.}

\textbf{a.}
\begin{align*}
  \int_{-1}^0kx^2&=1\\
  \frac{kx^3}{3}\biggr\rvert^0_{-1}&=1\\
  \frac{k}{3}&=1\\
  k&=3
\end{align*}

\textbf{b.}
\[E(X)=\int_{-1}^0x3x^2=\int_{-1}^03x^3=-0.75\]
\[E(X^2)=\int_{-1}^0x^23x^2=\int_{-1}^03x^4=0.6\]
\[Var(X)=E(X^2)-E(X)^2=0.6-(-0.75)^2=0.0375\]

\textbf{c.}
\[F(x)=\int_{-1}^x3x^2=x^3+1\]

\textbf{d.}
\[F(x)=x^3+1=0.5 \implies x^3=-0.5 \implies x=-0.7937\]

\textbf{4.}
Es la probabilidad de que la funcion de densidad de A sea mayor a 1/3

\[\int_{\frac{1}{3}}^1 4(1-x)^3 dx=\frac{16}{81}\]

\textbf{5.}
\[f(x)=
  \begin{cases}
    \frac{1}{b-a}=\frac{1}{3}&1 \le x \le 4\\
    0&sino
  \end{cases}\]
\[F(x)=
  \begin{cases}
    0&x<1\\
    \frac{x-1}{3}&1 \le x \le 4\\
    1&x>4
  \end{cases}\]
\[P(X>2)=1-F(2)=1-\frac{1}{3}=\frac{2}{3}\]
\[P(2<X<3)=F(3)-F(2)=\frac{2}{3}-\frac{1}{3}=\frac{1}{3}\]
\[P(X<1.5)=F(1.5)=\frac{1}{6}\]

\textbf{6.}
\[f(x)=
  \begin{cases}
    \frac{1}{5}&20 \le x \le 25\\
    0&sino
  \end{cases}\]
\[F(x)=
  \begin{cases}
    0&x<20\\
    \frac{x-20}{5}&20 \le x \le 25\\
    1&x>25
  \end{cases}\]

\textbf{a.}

$P(X \le 23)=F(23)=\frac{3}{5}$

\textbf{b.}

$E(X)=\frac{b+a}{2}=\frac{45}{4}$

\textbf{7.}
\[f(x)=
  \begin{cases}
    0.01*e^{-0.01x}&x>0\\
    0&x<0
  \end{cases}\]
\[F(x)=
  \begin{cases}
    1-e^{-0.01x}&x>0\\
    0&x<0
  \end{cases}\]
\[E(X)=\frac{1}{\alpha}=100\]

\textbf{a.}

$P(X<E(X))=P(X<100)=F(100)=1-e^{-0.01*100}=1-e^{-1}$

\textbf{b.}

$F(X)=0.5 \implies 1-e^{-0.01x}=0.5 \implies e^{-0.01x}=0.5 \implies -0.01x=ln(0.5)$

$x=69.3147$

\textbf{c.}

$P(X>200)=1-F(200)=1-(1-e^{-0.01*200})=e^{-2}$

se define

$Y: \textrm{cantidad de componentes entre 3 que duran mas de 200 horas}, Y \sim Bin(3,e^{-2})$
\begin{align*}
  P(Y \ge 2)&=P(Y=2)+P(Y=3)\\
            &=\binom{3}{2}(e^{-2})^2(1-e^{-2})+(e^{-2})^3\\
            &=3e^{-4}(1-e^{-2})+e^{-6}\\
            &=0.0476
\end{align*}

\textbf{8.}

$X_t: \textrm{numero de accidentes en t dias} \sim Po(\frac{2t}{7})$

$Y: \textrm{dias que pasan entre dos accidentes} \sim Exp(\frac{2}{7})$

\textbf{a.}
\[P(Y>3)=1-F_Y(3)=1-(1-e^{-\frac{2*3}{7}})=e^{-\frac{6}{7}}\]

\textbf{b.}
$Z: \textrm{cantidad de dias hasta el tercer accidente} \sim Erl(\frac{2}{7},3)$
\[P(Z>12)=1-F_z(12)=1-\frac{\gamma(3,12*\frac{2}{7})}{2!}=1-\frac{\gamma(3,\frac{24}{7})}{2!}\]

\textbf{10.}
\begin{align*}
  P(-a<Z<a)&=2*P(0<z<a)\tag{Z simetrica alrededor de 0}\\
           &=2*(F(a)-F(0))\\
           &=2F(a)-2F(0)\\
           &=2F(a)-2*0.5\tag{$\mu=0$ es la mediana por Z normal}\\
           &=2F(a)-1\\
           &=2P(Z<a)-1
\end{align*}

\textbf{11.}
\[Z=\frac{X-500}{50} \sim N(0,1) \implies X=50Z+500\]

\textbf{a.}
\[P(X>580)=P(50Z+500>580)=P(Z>\frac{8}{5})=1-F_Z(\frac{8}{5})=1-0.9452=0.0548\]

\textbf{b.}
\[P(X<450)=P(50Z+500<450)=P(Z<-1)=F_Z(-1)=0.1587\]

\textbf{12.}
\[Z=\frac{X-0.13}{0.005} \sim N(0,1) \implies X=0.005X+0.13\]
\begin{align*}
  P(0.12<X<0.14)&=P(\frac{0.12-0.13}{0.005}<Z<\frac{0.14-0.13}{0.005})\\
                &=2F_Z(2)-1\tag{teorema ejercicio 10}\\
                &=2*0.9772-1\\
                &=0.9545
\end{align*}
\[Y: \textrm{cantidad de 4 que cumplen con las especificaciones} \sim Bin(4,0.9545)\]
\[P(Y=4)=0.9545^4=0.8301\]

\textbf{13.}

\textbf{a.}
\[P_{D_1}(D_1>45)=P_Z(Z>\frac{5}{36})=1-F_Z(\frac{5}{36})=0.4443\]
\[P_{D_2}(D_2>45)=P_Z(Z>\frac{0}{9})=1-F_Z(0)=1-0.5=0.5\]

Conviene mas el segundo producto

\textbf{b.}
\[P_{D_1}(D_1>48)=P_Z(Z>\frac{8}{36})=1-F_Z(\frac{8}{36})=1-0.5871=0.4129\]
\[P_{D_2}(D_2>48)=P_Z(Z>\frac{3}{9})=1-F_Z(\frac{3}{9})=1-0.6293=0.3707\]

Conviene mas el primero

\end{document}