\documentclass[12pt,fleqn]{article}
\usepackage{fullpage}
\usepackage{amsmath}
\usepackage{multirow}
\usepackage{indentfirst}
\usepackage{diagbox}
\title{\LARGE \textbf{Trabajo práctico: unidad 6}\\\large Probabilidad y estadística}
\author{Martín Rossi}
\date{}
\begin{document}
\maketitle
\subsection*{1.}
$X_i: \textrm{'resultado en la apuesta i' para } i=1...50 \textrm{ con distribuciones:}$

$P_{X_i}(5c)=0.1$

$P_{X_i}(-c)=0.9$

Entonces para $i=1...50$:

$E(X_i)=5c*0.1+(-c)*0.9=-0.4c$

$E(X_i^2)=(5c)^2*0.1+(-c)^2*0.9=25c^2*0.1+c^2*0.9=3.4c^2$

$V(X_i)=E(X_i^2)-E(X_i)^2=3.4c^2-(-0.4c)^2=3.4c^2-0.16c^2=3.24c^2$\\

Ahora defino la variable aleatoria:

$S=X_1+X_2+...+X_{50}$

Lo que hay que calcular es $P(S \ge 0)$.

Como las $X_i$ son independientes, por el teorema central del límite S es aproximadamente normal y sus parámetros son:

$\mu=E(X_1)+E(X_2)+...+E(X_{50})=50*(-0.4c)=-20c$

$\sigma=\sqrt{V(X_1)+V(X_2)+...+V(X_{50})}=\sqrt{50*3.24c^2}=12.7279c$\\

Si defino $Z=\frac{S-\mu}{\sigma}=\frac{S+20c}{12.7279c}, Z \sim N(0,1)$
\begin{align*}
  P(S \ge 0)&=P(\frac{S+20c}{12.7279c}\ge\frac{0+20c}{12.7279c})\\
            &=P(Z\ge 1.5714)\\
            &=1-P(Z<1.5714)\\
            &\approx 1-0.9418\\
            &=0.0582
\end{align*}

Por lo tanto la probabilidad de que después de 50 apuestas no se pierda dinero es aproximadamente 5.82\%.
\newpage
\subsection*{2.}
$R: \textrm{'longitud del recipiente en cm'}, R \sim N(\mu=65cm,\sigma=0.5cm)$

$A: \textrm{'longitud de un bloque tipo A en cm'}, A \sim N(\mu=1.95cm,\sigma=0.01cm)$

$B: \textrm{'longitud de un bloque tipo B en cm'}, B \sim N(\mu=0.83cm,\sigma=0.02cm)$\\

Y defino las nuevas variables aleatorias:

$S=20*A+30*B$ (porque un ensamble tiene 20 bloques A y 30 bloques B)

$X=S-R$\\

Como S es combinación lineal de variables aleatorias independientes con distribución normal, S también es normal con parámetros:

$\mu=20*1.95cm+30*0.83cm=63.9cm$

$\sigma=\sqrt{20^2*(0.01cm)^2+30^2*(0.02cm)^2}=\sqrt{0.04cm^2+0.36cm^2}=\sqrt{0.4cm^2}=0.6325cm$\\

X es una resta de variables aleatorias normales independientes, X también lo es y sus parámetros son:

$\mu=63.9cm-65cm=-1.1cm$

$\sigma=\sqrt{(0.6325cm)^2+(0.5cm)^2}=\sqrt{0.4cm^2+0.25cm^2}=\sqrt{0.65cm^2}=0.8062cm$

Lo que hay que calcular es $P(S\le R)\iff P(S-R\le 0) \iff P(X\le 0)$.\\

Si ahora defino $Z=\frac{X+1.1cm}{0.8062cm}, Z \sim N(0,1)$.
\begin{align*}
  P_X(X\le 0)&=P_X(\frac{X+1.1cm}{0.8062cm}\le\frac{1.1cm}{0.8062cm})\\
             &=P_Z(Z\le 1.3644)\tag{Por la probabilidad de sucesos equivalentes}\\
             &\approx 0.9131
\end{align*}

El ensamble entrará en el recipiente con una probabilidad de aproximadamente 91.31\%.
\newpage
\subsection*{3.}
\subsubsection*{a.}
\begin{tabular}{|r|r|r|r|r|}
  \hline
  \diagbox{X}{Y}&0&1&2&$P_x(x)$\\\hline
  0&0.25&0.35&0.1&0.7\\\hline
  1&0.15&0.05&0.1&0.3\\\hline
  $P_y(y)$&0.4&0.4&0.2&1\\\hline
\end{tabular}\\

$\mathbf{P_y(Y=2)=0.2}$\\

$\mathbf{P(X=1,Y=0)=0.15}$\\

$0.3=E(X)=0*P_x(X=0)+1*P_x(X=1)=P_x(X=1)\iff \mathbf{P_x(X=1)=0.3}$\\

$1=P_x(X=0)+P_x(X=1)=P_x(X=0)+0.3\iff \mathbf{P_x(X=0)=0.7}$\\

$0.8=E(Y)=0*P_y(Y=0)+1*P_y(Y=1)+2*P_y(Y=2)=P_y(Y=1)+2*0.2\iff$

$\mathbf{P_y(Y=1)=0.4}$\\

$1=P_y(Y=0)+P_y(Y=1)+P_y(Y=2)=P_y(Y=0)+0.4+0.2\iff \mathbf{P_y(Y=0)=0.4}$\\

$0.5=P(Y=1|X=0)=\frac{P(Y=1,X=0)}{P_x(X=0)}\iff \mathbf{P(Y=1,X=0)=0.5*0.7=0.35}$\\

$0.4=P_y(Y=1)=P(X=0,Y=1)+P(X=1,Y=1)=0.35+P(X=1,Y=1)\iff \mathbf{P(X=1,Y=1)=0.05}$\\

$0.4=P_y(Y=0)=P(X=0,Y=0)+P(X=1,Y=0)=P(X=0,Y=0)+0.15\iff \mathbf{P(X=0,Y=0)=0.25}$\\

$0.7=P_x(X=0)=P(X=0,Y=0)+P(X=0,Y=1)+P(X=0,Y=2)=0.25+0.35+P(X=0,Y=2)\iff \mathbf{P(X=0,Y=2)=0.1}$\\

$0.3=P_x(X=1)=P(X=1,Y=0)+P(X=1,Y=1)+P(X=1,Y=2)=0.15+0.05+P(X=0,Y=2)\iff \mathbf{P(X=0,Y=2)=0.1}$\\

Verificación de las varianzas:\\

$0.21=V(X)=E(X^2)-E(X)^2=0.3-0.3^2=0.21$\\

$0.56=V(Y)=E(Y^2)-E(Y)^2=(0.4+4*0.2)-0.8^2=0.56$
\newpage
\subsubsection*{b.}
Para ver si las variables son independientes se comprueba $P(X=x,Y=y)=P_x(x)P_y(y),\forall x,y$:\\

$P(X=0,Y=0)=0.25$

$P_x(0)P_y(0)=0.4*0.7=0.28$

Por lo tanto las variables no son independientes.\\
\begin{align*}
  \rho&=\frac{Cov(X,Y)}{\sqrt{V(X)V(Y)}}\\
      &=\frac{\sum_{(x,y)\in S}P(x,y)(x-0.3)(y-0.8)}{0.3429}\tag{Covarianza de variables discretas}\\
      &=\frac{0.01}{0.3429}\\
      &=0.02916
\end{align*}
\subsubsection*{c.}
$P(Y=2|X=0)=\frac{P(Y=2,X=0)}{P_x(X=0)}=\frac{0.1}{0.7}=0.1429$

Dado que no hay defectos de tipo D1 en una pieza, hay un 14.29\% de probabilidades de que haya 2 defectos de tipo D2 en la misma.
\subsubsection*{d.}
$C: \textrm{'costo de reparación en una pieza'}$

$C=3*X+4*Y$
\begin{align*}
  E(C)&=E(3*X+4*Y)\\
      &=3*E(X)+4*E(Y)\tag{1}\\
      &=3*0.3+4*0.8\\
      &=4.1
\end{align*}
\begin{align*}
  V(C)&=V(3*X+4*Y)\\
      &=3^2*V(X)+4^2*V(Y)\tag{2}\\
      &=9*0.21+16*0.56\\
      &=10.85
\end{align*}

(1)Esperanza de una combinación lineal.

(2)Varianza de una combinación lineal, porque las variables X e Y son independientes.
\end{document}