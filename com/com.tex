% Created 2021-10-02 sáb 22:08
% Intended LaTeX compiler: pdflatex
\documentclass[11pt]{article}
\usepackage[utf8]{inputenc}
\usepackage[T1]{fontenc}
\usepackage{graphicx}
\usepackage{grffile}
\usepackage{longtable}
\usepackage{wrapfig}
\usepackage{rotating}
\usepackage[normalem]{ulem}
\usepackage{amsmath}
\usepackage{textcomp}
\usepackage{amssymb}
\usepackage{capt-of}
\usepackage{hyperref}
\usepackage{fullpage}
\date{\today}
\title{}
\hypersetup{
 pdfauthor={},
 pdftitle={},
 pdfkeywords={},
 pdfsubject={},
 pdfcreator={Emacs 27.2 (Org mode 9.4.4)}, 
 pdflang={English}}
\begin{document}

\tableofcontents

\section{introduccion}
\label{sec:orgc524427}
\subsection{hardware}
\label{sec:org8c5edc2}
\subsection{sofware de red}
\label{sec:org30a9290}
\subsubsection{jerarquia de protocolos}
\label{sec:org76cd0a8}
\begin{itemize}
\item organizacion por capas. cada capa tiene una funcion diferenciada e independiente
\item intercambio de mensajes segun el protocolo de cada capa
\item en realidad los mensajes bajan hasta la capa inferior (medio fisico), donde se realiza la comunicacion
\item interfaz bien definida para comunicacion entre capas
\item arquitectura de red: conjunto de capas y protocolos
\item pila de protocolos: lista de protocolos usados por una arquitectura
\end{itemize}
\subsubsection{aspectos de diseño para cada capa}
\label{sec:org1ddbfd5}
\begin{itemize}
\item codigos de deteccion (y posible correccion) de errores
\item enrutamiento: eleccion de una ruta para enviar informacion
\item distribucion de protocolos en capas
\item mecanismos para embalar, desembalar y transmitir
\item escalabilidad
\item asignacion eficiente de recursos
\item uso del ancho de banda (multiplexado estadistico, fraccion fija)
\item control de flujo
\item confidencialidad, autenticacion e integridad
\end{itemize}
\subsubsection{tipos de servicios}
\label{sec:orgee96dfc}
\begin{enumerate}
\item orientados a la conexion
\label{sec:org051c6f2}
\begin{itemize}
\item se establece la conexion, se usa y se libera
\item en la mayoria de los casos se preserva el orden
\item como una linea telefonica
\end{itemize}
\item no orientados a la conexion
\label{sec:org07ca53e}
\begin{itemize}
\item cada mensaje lleva la direccion de destino completa
\item cada mensaje es enrutado en forma independiente
\item como el sistema postal
\end{itemize}
\item confiables
\label{sec:orgcdfd0cf}
\begin{itemize}
\item nunca pierden datos
\item acuse de recibo
\item introduccion de sobrecarga y retardos
\end{itemize}
\item no confiables
\label{sec:orga19d1da}
\begin{center}
\begin{tabular}{lll}
 & confiable & no confiable\\
conexion & secuencia de mensajes & voz sobre ip\\
 & flujo de bytes & \\
no conexion & mensajes de texto & mails\\
\end{tabular}
\end{center}
\end{enumerate}
\subsubsection{relacion entre servicios y protocolos}
\label{sec:org7f1c3de}
\begin{itemize}
\item un servicio se define como un conjunto de primitivas que una capa proporciona a la que esta encima de ella
\item el servicio define el que pero no el como
\item protocolo son las reglas de formato y significado de los paquetes o mensajes que se intercambian en la misma capa
\item servicio se relaciona con las interfaces entre capas
\item protocolo se relaciona con los paquetes que se envian entre distintas maquinas
\end{itemize}
\subsection{modelos de referencia}
\label{sec:org7445131}
\subsubsection{modelo osi}
\label{sec:orga279fb2}
\begin{enumerate}
\item capa fisica
\label{sec:org4b8b7b0}
\begin{itemize}
\item transmision de bits puros a traves de un canal de transmision
\item busca que lleguen los mismos bits que salieron
\item señales electricas para representar un bit
\item como se establece y se termina una comunicacion
\end{itemize}
\item capa de enlace de datos
\label{sec:org03738ec}
\begin{itemize}
\item transforma los bits puros en una linea que este libre de errores para la capa de red
\item divide los datos en tramos
\item control de transmision para emisores rapidos y receptores lentos
\end{itemize}
\item capa de red
\label{sec:org6431b5c}
\begin{itemize}
\item como se encaminan los paquetes del origen al destino
\item las rutas se basan en tablas estaticas o dinamicas
\item manejo de congestion
\item solucionar problemas para conectar redes heterogeneas
\end{itemize}
\item capa de transporte
\label{sec:orgf440571}
\begin{itemize}
\item aceptar datos de la capa superior, dividirlos en unidades mas pequeñas, pasar los datos a la capa de red y asegurar que las piezas lleguen al otro extremo
\item es una verdadera capa de extremo a extremo, a diferencia de las mas bajas
\end{itemize}
\item capa de sesion
\label{sec:org818a63f}
\begin{itemize}
\item control de dialogo
\item manejo de tokens
\item sincronizacion
\end{itemize}
\item capa de presentacion
\label{sec:orge92ad65}
\begin{itemize}
\item se enfoca en la sintaxis y la semantica de la informacion transmitida
\item maneja estructuras abstractas para intercambiar datos entre computadoras con diferentes representaciones de datos
\end{itemize}
\item capa de aplicacion
\label{sec:orgcf1b8de}
\begin{itemize}
\item protocolos que los usuarios necesitan
\end{itemize}
\end{enumerate}
\subsubsection{modelo tcp/ip}
\label{sec:orga63b07c}
\begin{enumerate}
\item capa de enlace
\label{sec:org5c55c33}
\begin{itemize}
\item capa sin conexion que opera a traves de distintas redes
\item describe que enlaces se deben llevar a cabo para cumplir con las necesidades de esta capa
\end{itemize}
\item capa de interred
\label{sec:orgd811ed6}
\begin{itemize}
\item permite que los host inyecten paquetes en cualquier red y que viajen independientemente a su destino
\item analogo al sistema de correo
\item define un formato de paquete y un protocolo oficial llamado ip y uno complementario llamado icmp
\item el ruteo de paquetes es el principal aspecto, y la congestion
\end{itemize}
\item capa de transporte
\label{sec:org36dfce3}
\begin{itemize}
\item permite que entidades en la misma capa mantengan una conversacion
\item tcp, udp
\end{itemize}
\item capa de aplicacion
\label{sec:orgda40a6c}
\begin{itemize}
\item reemplaza las capas de presentacion, sesion y aplicacion del modelo osi
\item telnet, ftp, smtp, dns, http
\end{itemize}
\end{enumerate}
\subsubsection{comparacion tcp/ip osi}
\label{sec:org090ab77}
\begin{itemize}
\item osi fue inventado antes que los protocolos, por eso es mas general. pero los diseñadores no sabian que funcionalidades colocar en cada capa
\item con tcp/ip paso al reves. los protocolos encajaron perfectamente, pero no era util para describir redes que no fueran tcp/ip
\item osi tiene 7 capas, tcp/ip tiene 4
\end{itemize}
\subsubsection{defectos de osi}
\label{sec:org65688c1}
\begin{itemize}
\item mala sicronizacion: para cuando se desarrollaron los protocolos osi, tcp/ip ya se estaba usando lo suficiente como para que los distribuidores no quisieran apoyar otra pila
\item mala tecnologia: el modelo es muy complejo. las capas de sesion y presentacion estan casi vacias, las de red y enlace llenas. son dificiles de implementar e ineficientes.
\item malas implementaciones: por su complejidad las primeras implementaciones eran lentas y pesadas. despues mejoraron pero la imagen quedo
\item malas politicas: osi se asocio con el gobierno estadounidense y tcp/ip con unix
\end{itemize}
\subsubsection{defectos de tcp/ip}
\label{sec:org1ebc62d}
\begin{itemize}
\item no se diferencian bien los conceptos de servicio, interfaz y protocolo
\item el modelo no es para nada general
\item la capa de enlace no es una capa sino una interfaz
\item no distingue la capa de enlace y la fisica
\end{itemize}
\section{capa fisica}
\label{sec:org96c1662}
\subsection{conceptos}
\label{sec:org4865990}
\begin{itemize}
\item serie de fourier
\item ancho de banda
\item banda base, pasa-banda
\item teorema de nyquist, teorema de shannon
\item relacion señal ruido S/N
\end{itemize}
\subsection{medios de transmision guiados}
\label{sec:org04b5302}
\subsubsection{medios magneticos}
\label{sec:orgcab5b4b}
\begin{itemize}
\item guardar la informacion en una cinta o medio removible y mandarlo fisicamente
\item \emph{nunca subestime el ancho de banda de una camioneta repleta de cintas que viaje a toda velocidad por la carretera}
\end{itemize}
\subsubsection{par trenzado}
\label{sec:orgba93fac}
\begin{itemize}
\item dos cables de cobre aislados
\item trenzados porque en paralelo forman una antena
\item la señal se transmite como la diferencia de voltaje entre los dos cables
\item el ruido afecta a los dos cables por igual, el diferencial se mantiene
\item sistema telefonico
\item informacion analogica o digital
\item el ancho de banda depende del grosor de los cables y la distancia. hasta varios mbps
\item ethernet usa cuatro, uno para cada direccion
\item hasta cat 6: utp (unshielded twisted pair). cat 7: stp
\end{itemize}
\subsubsection{cable coaxial}
\label{sec:orgfb5d25b}
\begin{itemize}
\item mejor blindaje y mayor ancho de banda que los tp, pero mas caro
\end{itemize}
\subsubsection{lineas electricas}
\label{sec:org6f9ed29}
\begin{itemize}
\item las compañias las han utilizado para comunicacion de baja velocidad
\item uso en el hogar para controlar dispositivos
\item dificil porque el cableado de las casas no esta hecho para enviar señales a alta frecuencia
\end{itemize}
\subsubsection{fibra optica}
\label{sec:org65705ba}
\begin{itemize}
\item lan, internet y ftth
\item un pulso de luz indica 1, la ausencia 0
\item cuando la luz pasa de un medio a otro (silice a aire) se refracta. el grado depende de los indices de refraccion de los medios. y para cualquier angulo mayor a un angulo critico la luz rebota completamente en el silice
\item fibra multimodal: varios rayos de luz en una fibra
\item fibra monomodo: un solo rayo de luz por fibra que es mucho mas angosta
\item tres bandas: 0.85 1.3 y 1.55 micras. anchos de banda de 25000 a 30000 ghz. la primera tiene mas atenuacion
\item fuentes: led y laser
\end{itemize}
\subsection{transmision inalambrica}
\label{sec:orga7a26e2}
\subsubsection{espectro electromagnetico}
\label{sec:org3e24383}
\begin{itemize}
\item los electrones se mueven y crean ondas electromagneticas
\item las ondas viajan siempre a la velocidad de la luz
\item \(\lambda f=c\)
\item espectro directo con salto de frecuencia: transmision dificil de detectar y bloquear. militares, bluetooth, versiones anteriores de 802.11
\item espectro directo de secuencia directa: multiples señales comparten ancho de banda. cdma, gps, 802.11b
\item uwb
\end{itemize}
\subsubsection{radiotransmision}
\label{sec:orgbbb6161}
\begin{itemize}
\item las ondas de radio son faciles de generar, recorren largas distancias y penetran edificios
\item son omnidireccionales
\item las propiedades dependen de la frecuencia. baja frecuencia: cruzan obstaculos pero se reduce la potencia rapidamente. alta frecuencia: viajan en linea recta y rebotan en obstaculos
\item ondas de alta frecuencia son absorbidas por la lluvia y otros obstaculos
\item como recorren grandes distancia la interferencia es un problema
\item estan reguladas por los gobiernos
\item vlf, lf y mf siguen la curvatura de la tierra. hf van en linea recta y rebotan en la ionosfera, tambien son absorbidas por la tierra
\end{itemize}
\subsubsection{transmision por microondas}
\label{sec:org80c7e36}
\begin{itemize}
\item relacion S/N alta, pero las antenas deben estar alineadas
\item microondas no atraviesan bien los edificios
\item comunicacion telefonica, celulares, television. lo que provoco escasez de espectro
\end{itemize}
\subsubsection{transmision infrarroja}
\label{sec:org9470818}
\begin{itemize}
\item comunicacion de corto alcance
\item no atraviesan objetos
\end{itemize}
\subsubsection{tranmision por ondas de luz}
\label{sec:org0a5c096}
\begin{itemize}
\item señalizacion optica mediante laser
\item gran ancho de banda a bajo costo y seguro. pero muy dificil de apuntar
\end{itemize}
\subsection{satelites de comunicacion}
\label{sec:org5a4cf36}
\begin{itemize}
\item un satelite es un enorme repetidor de microondas con varios transpondedores. transmite en modo \textbf{tublo doblado}
\item posicion de los satelites limitadas por el cinturon de van allen
\end{itemize}
\subsubsection{satelites geoestacionarios}
\label{sec:org152d359}
\begin{itemize}
\item satelites que orbitan a la misma velocidad de la que rota la tierra. parecen inmoviles desde el suelo
\item los primeros tenian un solo haz de luz que iluminaba la tierra, lo que se conoce como huella
\item actualmente tienen multiples haces que se enfocan en una pequeña area geografica. estos son los haces puntuales
\item vsat: terminales muy pequeñas que se utilizan para la transmision de tv
\item los vsat no se pueden comunicar entre ellos por su baja potencia. para ello usan de intermediario potentes estaciones en la tierra
\item aunque las señales viajen a la velocidad de la luz, dada las distancias tienen mas retardo que las comunicaciones terrestres
\item los satelites son medios de difusion por naturaleza
\end{itemize}
\subsubsection{ventajas de los satelites sobre la fibra optica}
\label{sec:org625d5c7}
\begin{itemize}
\item cuando se requiere un despliegue rapido, ganan los satelites
\item los satelites pueden enviar a cualquier parte del mundo
\item un mensaje que envia un satelite lo pueden recibir miles de estaciones al mismo tiempo
\end{itemize}
\subsection{modulacion digital y multiplexacion}
\label{sec:orga8adf27}
\begin{itemize}
\item modulacion digital: proceso de convertir bits en la señal que los representan
\item transmision en banda base: la señal ocupa una frecuencia desde 0 hasta un valor maximo que depende de la tasa de señalizacion. comun en cables
\item transmision pasa-banda: la señal ocupa una banda de frecuencias alrededor de la frecuencia de la señal portadora. comun en inalambrico y optico
\item multiplexacion: a compartir varias señales por un mismo canal
\end{itemize}
\subsubsection{transmision en banda base}
\label{sec:orgebee880}
\begin{itemize}
\item NRZ(non-return-to-zero): voltaje positivo para el 1 y uno nulo para el 0
\item el receptor muestrea a intervalos regulares y convierte de nuevo a bits. la señal no se vera igual a la que se envio por el ruido y el canal
\item eficiencia del ancho de banda
\begin{itemize}
\item con nrz la señal puede alternar entre positivo y negativo hasta cada 2 bits. necesita un ancho de banda B/2hz pasa tasa de B bps
\item una estategia es usar mas de 2 niveles de señalizacion. por ejemplo 4 voltajes para representar 2 bits a la vez como un simbolo
\item tasa de bits=tasa de simbolo*bits por simbolo
\item requiere una potencia mayor en el receptor para diferenciar los niveles
\end{itemize}
\item recuperacion del reloj
\begin{itemize}
\item el receptor debe saber cuando termina un simbolo y empieza otro
\item existe un limite en la precision de un reloj para muestrear señales
\item se podria enviar una señal del reloj por otra linea separada, pero seria mejor que si hubiera otra linea se usara para enviar datos
\item un truco seria usar xor entre las dos lineas para enviarlas en una sola. esta es la codificacion manchester y se usaba en ethernet clasico. lo malo es que requiere el doble de ancho de banda
\item una estrategia distinta es codificar los datos para que haya suficientes transiciones en la señal. ya que los problemas suceden en largas suceciones de 0 o 1
\item nrzi: 1 como una transicion y 0 como no hay transicion. usb usa este metodo. largas sucesiones de 1 no tienen problemas, pero de 0 si
\item 4b/5b: se asocian grupos de 4 bits a 5 bits segun una tabla fija, de manera que nunca haya tres 0 seguidos. agrega 25\% de sobrecarga. sobran 16 numeros de 5 bits, algunos se usan para control
\item para asegurar transiciones se puede hacer xor con una secuencia pseudoaleatoria. el receptor decodifica con la misma secuencia. esta debe ser facil de generar
\item pero la aleatorizacion no garantiza transiciones
\end{itemize}
\item señales balanceadas
\begin{itemize}
\item señales que tienen misma cantidad de voltajes positivos como negativos
\item ayuda a proveer transiciones para la recuperacion del reloj
\item codificacion bipolar: se alterna +1 y -1 voltios para el 1 y 0 voltios para el 0. en redes telefonicas ami
\item 8b/10b tambien para codigo balanceado
\end{itemize}
\end{itemize}
\subsubsection{transmision pasa-banda}
\label{sec:org7505eb5}
\begin{itemize}
\item en canales inalambricos no es practico usar rango de frecuencias que empiecen en 0
\item se puede tomar una señal en banda base que ocupe de 0 a b hz y desplazarla a otra pasa-banda que ocupe de s a s+b hz
\item se puede modular la amplitud (ask), la frecuencia (fsk) o la fase (psk)
\item psk puede ser bpsk (binaria) o qpsk (cuadratura)
\item se pueden combinar y usar mas niveles, comunmente amplitud y fase
\item diagrama de constelacion: forma de visualizar la modulacion combinada ask y psk. qpsk, qam-16, qam-64
\item simbolos adyacentes no deben diferir en muchos bits, porque serian mas suceptibles al ruido. para eso se usa codigo gray
\end{itemize}
\subsubsection{multiplexacion por division de frecuencia}
\label{sec:org0f95587}
\begin{itemize}
\item fdm: divide el espectro en bandas. cada usuario tiene posesion exclusiva de la banda
\item banda de guarda: exceso de banda que mantiene a los canales separados
\item ofdm: el ancho de banda del canal se divide en muchas subportadoras que envian de manera independiente. cada subportadora esta diseñada para ser 0 en el centro de las adyacentes. 802.11
\end{itemize}
\subsubsection{multiplexacion por division de tiempo}
\label{sec:orgb1d2900}
\begin{itemize}
\item tdm: los usuarios toman turnos y usan todo el ancho de banda, se toman los datos y se agregan al flujo agregado
\item para que funcione debe haber sincronizacion. se puede agregar tiempo de guarda
\end{itemize}
\subsubsection{multiplexacion por division de codigo}
\label{sec:org21505b3}
\begin{itemize}
\item cdm: forma de comunicacion de espectro diverso. una señal de banda estrecha se dispersa en una mas amplia. cdma
\item hace la señal mas tolerante a interferencias y permite que señales compartan la misma banda de frecuencia
\item cdma es extraer la señal deseada mientras lo demas se rechaza como ruido
\item cada tiempo de bit de divide en m intervalos llamados chips. en general 64 o 128 chips cada bit. a cada estacion se le asigna una secuencia de chip, un codigo de m bits. para transmitir un 1 envia la secuencia de chip, para el 0 la negacion
\item todas las secuencias de chip son ortogonales por pares
\item si varias estaciones envian al mismo tiempo se suman
\end{itemize}
\section{capa de enlace}
\label{sec:org03e8fa8}
\subsection{cuestiones de diseño}
\label{sec:org4764a79}
\begin{itemize}
\item funciones: dar a la capa de red una interfaz de servicios bien definida. manejar errores. controlar flujo
\item toma los datos que obtiene de la capa de red y los encapsula en tramas
\end{itemize}
\subsubsection{servicios dados a la capa de red}
\label{sec:org7bf062e}
\begin{itemize}
\item transferir datos de la maquina de origen a la de destino
\item 3 servicios razonables
\begin{itemize}
\item sin conexion ni confirmacion de recepcion: tasa de error baja. trafico en tiempo real. ethernet
\item sin conexion con confirmacion: canales no confiables. 802.11 (wifi)
\item con conexion y confirmacion: cada trama esta enumerada. se garantiza que lleguen solo una vez y en orden. canales largos y no confiables. satelites y red telefonia larga
\end{itemize}
\end{itemize}
\subsubsection{entramado}
\label{sec:org4a1f258}
\begin{itemize}
\item la capa fisica no garantiza que el flujo de bits este libre de errores
\item un metodo es dividir el flujo en tramas discretas y agregarles una suma de verificacion
\item division de tramas
\begin{itemize}
\item conteo de bytes: agrega en el encabezado la cantidad de bytes en la trama. si se altera este valor se pierde la sincronia. rara vez se usa solo
\item bytes bandera con relleno de bytes: cada trama inicia y termina con bytes especiales. si aparece la bandera en los datos se antecede un escape. y si aparece un escape se pone otro escape adelante. simplificacion de ppp
\item bits bandera con relleno de bits: igual a bytes pero sin la restriccion de 1 byte=8 bits. hdlc. usb. se usan 6 bits en 1 para delimitar. cada vez que se ven 5 bits en 1 se agrega un 0
\item violaciones de codificacion de la capa fisica: si se usa por ejemplo 4b/5b en la capa fisica se pueden usar los codigos no utilizados para el inicio y fin de trama
\end{itemize}
\end{itemize}
\subsubsection{control de errores}
\label{sec:orga6a04f5}
\begin{itemize}
\item asegurar la entrega de datos confiable: retroalimentacion al emisor de lo que esta ocurriendo del otro lado. positiva y negativa
\item puede desaparecer la trama por completo, o la de retroalimentacion. para eso tambien se usan temporizadores para enviar nuevamente
\item ahora puede que se reciba la misma trama dos veces. para eso se usan numeros de secuencia
\end{itemize}
\subsubsection{control de flujo}
\label{sec:org67754c5}
\begin{itemize}
\item que hacer cuando un emisor envia mas tramas de las que el receptor puede aceptar. ejemplo telefono y sitio web
\item control de flujo basado en retroalimentacion: el receptor envia cuando puede aceptar mas datos
\item control de flujo basado en tasa: el protocolo tiene un mecanismo integrado que limita la tasa de envio
\end{itemize}
\subsection{deteccion y correccion de errores}
\label{sec:orge35c969}
\begin{itemize}
\item estategia: incluir redundancia en los datos.
\item codigo de correccion de errores: para que el receptor pueda deducir que datos se quisieron enviar. fec
\item codigo de deteccion de errores: para que sepa que hubo un error pero nada mas y solicite retransmision
\item en fibra optica conviene la deteccion porque es rapido reenviar. en canales inalambricos es mejor correccion
\item los bits de redundancia tambien pueden llegar mal. asi que nunca se podran manejar todos los errores
\item los errores en rafaga tienen sus ventajas y desventajas
\end{itemize}
\subsection{protocolos de enlace de datos}
\label{sec:orgaf73b4b}
\subsubsection{paquetes sobre sonet}
\label{sec:org706cab9}
\begin{itemize}
\item sonet se utiliza sobre canales de fibra optica de area amplia
\item ppp se usa para diferenciar paquetes ocasionales del flujo continuo en el que se transportan
\end{itemize}
\subsubsection{ppp}
\label{sec:orga8028ed}
\begin{itemize}
\item ppp orientado a bytes, hdlc a bits
\item metodo de entramado sin ambiguedades, tambien maneja deteccion de errores
\item protocolo para activar lineas, probarlas, negociar y desactivarlas. lcp
\item mecanismo para negociar opciones de capa de red independientemente del protocolo de red usado
\item uso de banderas como delimitacion y bytes de escape
\item la carga util se mezcla aleatoriamente antes de insertarla en sonet para garantizar mas transiciones que necesita sonet
\item configuracion enlace ppp
\begin{itemize}
\item muerto
\item establecer (cuando hay conexion en la capa fisica): intercambio de paquetes lcp
\item autentificar (si lo anterior fue exitoso): se verifican identidades
\item red: paquetes ncp para configurar la capa de red
\item abrir: intercambio de datos
\item terminar
\end{itemize}
\end{itemize}
\section{subcapa control acceso al medio}
\label{sec:org628aca9}
\begin{itemize}
\item los enlaces de red pueden ser punto a punto o difusion
\item subcapa mac es la parte inferior de la de enlace de datos
\end{itemize}
\subsection{problema de asignacion de canal}
\label{sec:org4318e87}
\begin{itemize}
\item asignar un solo canal de difusion entre varios usuarios competidores
\end{itemize}
\subsubsection{asignacion estatica}
\label{sec:org4534f1f}
\begin{itemize}
\item dividir la capacidad mediante el uso de multiplexacion. cuando hay una pequeña cantidad de usuarios constantes
\item si varia el numero de emisores y ese numero es grande se vuelve ineficiente
\item lo mismo sucede con otras formas estaticas de dividir un canal
\end{itemize}
\subsubsection{supuestos para la asignacion dinamica}
\label{sec:org19dd478}
\begin{itemize}
\item trafico independiente: las estaciones son independientes
\item canal unico: hay un solo canal para todas las comunicaciones
\item colisiones observables: todas las estaciones pueden detectar colisiones. que seran enviadas luego
\item tiempo continuo o ranurado: se puede considerar de las dos maneras
\item deteccion de portadora o sin deteccion: si hay deteccion las estaciones pueden saber si el canal esta en uso. sino mandan y despues determinan si tuvo exito
\end{itemize}
\subsection{protocolos de acceso multiple}
\label{sec:org03b6069}
\subsubsection{aloha}
\label{sec:org0142893}
\begin{itemize}
\item aloha puro
\begin{itemize}
\item despues de enviar su trama a la computadora central, esta difunde la trama a todas las estaciones. asi el emisor sabe si llego su trama
\item si la trama fue destruida espera un tiempo aleatorio y manda de nuevo
\item cada vez que dos tramas intenten ocupar el canal al mismo tiempo habra colision, por mas que sea un solapamiento pequeño
\end{itemize}
\item aloha ranurado
\begin{itemize}
\item como el metodo puro pero el tiempo se divide en ranuras discretas
\item sincronizacion por medio de una estacion que emita una señal al comienzo de cada intervalo
\end{itemize}
\end{itemize}
\subsubsection{protocolos de acceso multiple con deteccion de portadora}
\label{sec:orgd6a2004}
\begin{itemize}
\item csma persistente-1
\begin{itemize}
\item la estacion escucha el canal para ver si alguien esta enviando, sino envia. si ocurre una colision espera y manda de nuevo
\item el retardo de propagacion tiene un efecto importante en las colisiones. esta posibilidad depende del numero de tramas que quepan, o producto de ancho de banda-retardo
\item en lan como el retardo es pequeño, no habra muchas colisiones
\end{itemize}
\item csma no persistente
\begin{itemize}
\item a diferencia del persistente-1 si el canal esta en uso espera un tiempo y repite el proceso. no se queda escuchando constantemente
\item mejor uso del canal pero mayor retardo
\end{itemize}
\item csma persistente-p
\begin{itemize}
\item para canales ranurados
\item si el canal esta inactivo, envia con probabilidad p y espera a la siguiente ranura con probabilidad 1-p
\end{itemize}
\item csma con deteccion de colisiones (csma/cd)
\begin{itemize}
\item base de la clasica ethernet
\item el hardware escucha a la vez que envia. si la señal que recibe es distinta a la que envia, esta ocurriendo una colision
\item periodos alternantes de contencion y transmision con periodos de inactividad que ocurriran cuando todas las estaciones esten en reposo
\end{itemize}
\end{itemize}
\subsubsection{protocolos libres de colisiones}
\label{sec:org7667dc2}
\begin{itemize}
\item protocolo de mapa de bits
\begin{itemize}
\item cada periodo de contencion consiste en n ranuras
\item las estaciones envian 1 si tienen tramas para enviar en ese periodo pero solo en su ranura correspondiente
\item luego cuando ya hay conocimiento de quien va a mandar mandan las tramas en orden
\item protocolos de revervacion
\end{itemize}
\item paso de token
\begin{itemize}
\item pasa un pequeño mensaje llamado token de una estacion a otra en un orden determinado. token ring
\item solo puede enviar la que tenga el token
\item cuando la estacion que envio recibe su misma trama la elimina para terminar el ciclo
\item no hace falta que sea un anillo. token bus
\end{itemize}
\item conteo descendente binario
\begin{itemize}
\item anteriores no escalan a redes con miles de estaciones
\item las estaciones que quieren usar el canal envian su direccion binaria y hacen or de todo lo que reciben
\item tan pronto como una estacion ve que una posicion de bit de orden alto, cuya direccion es 0, ha sido sobreescrita por un 1, se da por vencida
\end{itemize}
\end{itemize}
\subsubsection{protocolos de contencion limitada}
\label{sec:org927a335}
\begin{itemize}
\item en condicion de carga ligera es preferible contencion
\item al reves para libres de colision
\item protocolos de contencion limitada combinan los dos anteriores
\item protocolo de recorrido de arbol adaptable
\begin{itemize}
\item en la ranura 0 todas las estaciones intentan adquirir el canal. si una lo logra bien y sino se dividen en dos grupos y se va formando un arbol de decision
\item a mayor carga la busqueda debe iniciar mas abajo en el arbol
\end{itemize}
\end{itemize}
\subsubsection{protocolos de lan inalambrica}
\label{sec:org0440e2f}
\begin{itemize}
\item problema de la terminal oculta
\item problema de la terminal expuesta
\item maca (acceso multiple con prevencion de colisiones)
\begin{itemize}
\item el emisor estimula al receptor para que envie una trama corta. las estaciones cercanas tambien escuchan y evitan enviar a la vez
\item rts/cts
\item en caso de colision un transmisor espera un tiempo y vuelve a intentar de nuevo
\end{itemize}
\end{itemize}
\subsection{ethernet}
\label{sec:org76db40d}
\begin{itemize}
\item 802.3
\item ethernet clasica (visto hasta ahora) y conmutada (switches)
\end{itemize}
\subsubsection{capa fisica de ethernet clasica}
\label{sec:orge305de1}
\begin{itemize}
\item un solo cable de donde se conectaban todas las maquinas
\item ethernet gruesa 500m y 100 maquinas
\item ethernet delgada 185m y 30 maquinas
\item longitud maxima por segmento conectada con repetidores
\end{itemize}
\subsubsection{protocolo de subcapa mac para ethernet clasica}
\label{sec:orga8f9ec0}
\begin{itemize}
\item multidifusion (a un grupo de estaciones) y difusion (a todas)
\item direcciones globalmente unicas
\item el tipo especifica a que proceso darle la trama
\item campos tipo y longitud en conflicto. despues se usaron los dos: se interpreta segun si es mayor a la maxima longitud
\item tamaño de trama maximo y minimo. se puede rellenar
\item csma/cd
\begin{itemize}
\item tras una colision el tiempo se divide en ranuras discretas de longitud igual a la ida y vuelta para el peor caso del cable
\item retroceso exponencial binario: despues de la colision n cada estacion espera de 0 a 2\textsuperscript{n}-1 ranuras para enviar de nuevo
\end{itemize}
\end{itemize}
\subsubsection{ethernet conmutada}
\label{sec:org9e322c5}
\begin{itemize}
\item se empezaron a usar hubs en vez de un solo cable
\item las redes se podian saturar porque los hubs no incrementan la capacidad. de ahi se empezaron a usar los switch
\item los switches envian tramas solo a los puertos para los que estan destinadas
\item en un switch cada puerto es su dominio de colision independiente
\item si el cable es full duplex (comun) no hay colisiones. si es half duplex se usa csma/cd
\item en un hub las tramas se envian a todos, aumentando la probabilidad de intrusos
\item un switch puede tener conectado un hub, asi actua como un concentrador
\end{itemize}
\subsubsection{fast ethernet}
\label{sec:org0b21931}
\begin{itemize}
\item se mantuvo la ethernet anterior pero mas rapida
\item se permiten tres medios: par trenzado categoria 3 y 5, fibra optica
\item casi todos los switches pueden manejar 10mbps (anterior) o 100mbps (fast)
\end{itemize}
\subsubsection{gigabit ethernet}
\label{sec:orge245984}
\begin{itemize}
\item en half duplex se usa csma/cd, en full duplex no
\item con 1gbps una trama minima que es enviada no llegaria a recorrer el cable antes que termine de enviar, por eso de limito la longitud a 200m
\item extension de portadora: el hardware agrega datos para hacer la trama de 512 bytes. no hay que hacer cambios de software
\item rafaga de tramas: el emisor envia una secuencia de tramas concatenadas en una sola transmision. si hay suficientes tramas, es preferible a la extension de portadora
\item en la actualidad la mayoria de las interfaces ethernet soportan los 3 tipos
\end{itemize}
\subsubsection{10 gigabit ethernet}
\label{sec:org25b2d14}
\subsection{redes lan inalambricas}
\label{sec:org83509d3}
\begin{itemize}
\item medio de comunicacion ondas electromagneticas
\item tres tipos de redes: wpan, wlan, wwan
\item modelos basados en pila: osi, tcp/ip, otros
\end{itemize}
\subsubsection{wi-fi o wlan}
\label{sec:orgf59cef5}
\begin{itemize}
\item capa fisica y enlace de osi
\item 802.11
\item arquitectura celular: el sistema se subdivide en celdas. cada celda (bss) se controla por una estacion (ap)
\item la capa fisica
\begin{itemize}
\item funciones
\begin{itemize}
\item codificacion/decodificacion de las señales
\item generacion/remocion de cabeceras
\item transmision/recepcion de bits
\item especificaciones del medio de transmision
\end{itemize}
\item fhss(espectro disperso con salto de frecuencia): transmision en intervalos de tiempo a frecuencias distintas que el emisor y el receptor conocen. resistente al ruido y mas seguro
\item dsss(espectro disperso con secuencia directa): transmitir con una secuencia de bits de alta velocidad llamados chips. secuencia de barker
\item mimo(multiple entrada/multiple salida): aparatos con varias antenas para generar subcanales de transmision
\end{itemize}
\item capa de enlace
\begin{itemize}
\item funciones
\begin{itemize}
\item capa control acceso al medio
\begin{itemize}
\item transmision: ensamblado de datos en tramas con campos de direccionamiento y deteccion de errores
\item recepcion: desensamblado de tramas, reconocimiento de direcciones y deteccione de errores
\item administra acceso al medio de transmision
\end{itemize}
\item capa control de enlace logico
\begin{itemize}
\item interface a las capas superiores, control de errores y flujo
\end{itemize}
\end{itemize}
\item a diferencia de ethernet para wifi debe haber acuse de recibo
\item puede darle el problema de que una estacion no llegue a escuchar cuando otra en la misma red este mandando y se produzcan colisiones. estacion oculta
\item rts/cts
\item dcf: mecanismo basico de csma/ca. primero se verifica que nadie use el canal. las estaciones retardan aleatoriamente las tramas y luego escuchan para evitar colisiones. a veces usan rts/cts
\item pcf: tecnica de interrogacion circular desde el ap. servicios de tipo sincrono
\end{itemize}
\item funciones de deteccion de portadoras
\begin{itemize}
\item para deetrminar si el medio se encuentra disponible
\item dos tipos: de la capa fisica y deteccion de portadoras virtuales(nav)
\end{itemize}
\item espaciamiento intertrama: cuatro diferentes espaciamientos para diferentes prioridades
\item tres tipos de trama: datos, control y gestion
\item control de enlace logico
\begin{itemize}
\item direccionamiento de estaciones conectadas al medio y control de flujo
\item basado en el protocolo hdlc
\item 3 tipos de servicios: sin conexion y sin reconocimiento, con y sin, sin y con
\end{itemize}
\end{itemize}
\subsubsection{wpan}
\label{sec:org5548429}
\begin{itemize}
\item dispositivos perifericos
\item bluetooth, homerf, zigbee, infrarrojo
\item bluetooth
\begin{itemize}
\item clase 1, 2 y 3 segun la potencia
\item piconet
\begin{itemize}
\item un nodo maestro y hasta 7 nodos esclavos activos. hasta 255 en total
\item puede haber varias piconets conectadas de un nodo esclavo puente(scatternet)
\item capa fisica
\begin{itemize}
\item sistema de baja potencia. pocos metros
\item 79 canales de 1mhz. modulacion desplazamiento de frecuencia
\item misma banda que 802.11 pero es mas problable que bluetooth interfiera con 802.11 que al reves
\end{itemize}
\item capa banda base
\begin{itemize}
\item perecido a la capa mac
\item multiplexion por division de tiempo: el maestro transmite en ranuras pares y los esclavos en impares
\item enlace acl: capa l2cap. mejor esfuerzo
\item enlace sco: datos en tiempo real. se asigna una ranura fija a cada direccion. no se retransmiten datos
\end{itemize}
\item administrador de enlace
\item capa adaptacion y control de enlace logico(l2cap)
\begin{itemize}
\item acepta paquetes de capa superior y los divide en tramas
\item maneja la multiplexion
\item se encarga de la calidad de los requerimientos de servicio. establece enlaces, negocia el tamaño de carga util
\end{itemize}
\end{itemize}
\end{itemize}
\item bluetooth smart(ble)
\begin{itemize}
\item 40 canales de 2mhz
\item no es directamente compatible con el anterior. si en modo dual(smart ready)
\item topologia broadcasting
\begin{itemize}
\item enviar datos a cualquier dispositivo que este escuchando el medio
\item envia periodicamente paquetes de anuncio por canales especificos
\end{itemize}
\item topologia conexiones
\begin{itemize}
\item conexion permanente y periodicamente se intercambian datos entre maestro y esclavo
\end{itemize}
\item un dispositivo puede ser maestro y esclavo. un maestro puede ser conectado a multiples esclavos. un esclavo a multiples maestros
\item perfiles genericos: perfil de acceso generico(gap), perfil de atributo generico(gatt)
\item capa de enlace
\begin{itemize}
\item varios estados
\begin{itemize}
\item espera: no transmite ni recibe. modo ahorro
\item anuncio: un esclavo envia paquetes en canales de anuncio. recibe tambien desde un maestro
\item exploracion: escucha los paquetes de anuncio que envian los dispositivos
\item inicializacion: usado por el maestro antes de iniciar una conexion. escucha hasta que recibe el anuncio de un esclavo deseado y se conecta
\end{itemize}
\end{itemize}
\end{itemize}
\end{itemize}
\subsubsection{sistema de telefonia y comunicaciones moviles}
\label{sec:org6e2509a}
\begin{itemize}
\item division celular: dividir en zonas pequeñas donde se reutilizan canales disponibles
\item reutilizacion de frecuencias
\begin{itemize}
\item se asigna a cada celda un grupo de frecuencias, de modo que no se compartan con celdas vecinas
\item el grupo de celdas que no comparten canales se llama cluster
\end{itemize}
\item modo de funcionamiento
\begin{itemize}
\item simplex: no se puede transmitir y recibir simultaneamente por enlaces de subida y bajada
\item duplex: los dos enlaces usan portadoras distintas y se pueden usar a la vez
\end{itemize}
\item desde 1g hasta 4g+. 5g sin estandarizar
\item arquitectura
\begin{itemize}
\item equipo de usuario: contiene una tarjeta que le permita usar el servicio. se conecta a traves de una interfaz de radio
\item red de acceso: sustenta la transmision de radio con los usuarios para conectarlos con la red troncal
\item red troncal: control de acceso, gestion de movilidad, gestion de sesiones de datos, etc
\end{itemize}
\item tipos de redes de acceso: gerand/utran(3g) y e-utran(lte)
\item la red troncal se divide en tres
\begin{itemize}
\item dominio de circuitos: todas las entidades que dan servicios basados en conmutacion de circuitos. accesible a traves de geran y utran, e-utran no usa
\item dominio de paquetes: basado en conmutacion de paquetes. dos implementaciones: gprs y epc. gprs fue la primera en contexto de las redes anteriores. epc es la nueva de lte
\item subsistema ims: provision de servicios ip basados en el protocolo sip. asociada a lo multimedia y utiliza servicios del dominio de paquetes
\end{itemize}
\item arquitectura de lte
\begin{itemize}
\item eps(evolved packet system), enteramente basada en paquetes ip, tanto servicios en tiempo real como transmision de datos
\item los componentes son: la red e-utran, el dominio de paquetes epc y el sistema ims
\item contempla el acceso al servicio de redes utran y geran, y otras redes que no pertenecen a la misma familia
\item la red de acceso se compone de una unica entidad enb, que proporciona conectividad entre usuarios y la red troncal
\item enb usa tres interfaces: e-utran uu(usuarios-enb), s1(enb-troncal) y x2(enb-enb)
\end{itemize}
\item capa fisica
\begin{itemize}
\item ofdma para enlace descendente y sc-fdma para ascendente
\item qpsk, 16qam y 64qam descendente, qpsk, 64qam ascendente
\end{itemize}
\item interfaz de radio
\begin{itemize}
\item tres tipos de transferencia: difusion de señalizacion de control, envio de paquetes ip y transferencia de señalizacion de control
\end{itemize}
\item ofdma
\begin{itemize}
\item diversidad multiusuario: la asignacion de subportadoras se realizan dinamicamente
\item diversidad frecuencial: es posible asignar al usuario subportadoras no contiguas, suficientemente separadas
\item robustez en la propagacion multicamino: fuerte a la interferencia intersimbolica por la propagacion por multiples caminos
\item flexibilidad de banda asignada: permite acomodar las velocidades a usuarios segun lo que requieran
\item granularidad en recursos asignables: para acomodar servicios con diferente calidad
\item elevada relacion entre potencia media e instantanea
\item suceptibilidad a errores de frecuencia: cuando hay desplazamientos de frecuencia hay interferencias. se requieren mecanismos de sincronizacion
\end{itemize}
\item sc-fdma
\begin{itemize}
\item variaciones reducidas entre potencia media e instantanea
\item posibilidad de llevar a cabo de forma sencilla mecanismos de ecualizacion en el dominio de la frecuencia
\item capacidad de proporcionar asignacion de banda flexible
\end{itemize}
\end{itemize}
\subsection{conmutacion de la capa de enlace de datos}
\label{sec:orge89492e}
\begin{itemize}
\item lan de lanes con puentes
\end{itemize}
\subsubsection{usos de puentes}
\label{sec:orgbf0dbf8}
\begin{itemize}
\item universidades y departamentos tienen sus propias redes lan separadas, pero tambien requieren comunicarse entre ellas
\item la organizacion puede estar separada geograficamente
\item dividir una sola red lan en varias para alivianar la carga
\item dos algoritmos para que los puentes sean transparentes
\end{itemize}
\subsubsection{puentes de aprendizaje}
\label{sec:org90db267}
\begin{itemize}
\item cada puerto del switch define un dominio de colision
\item si una estacion se quiere comunicar con otra dentro del mismo segmento el switch debe descartar las tramas porque no es necesario reenviarlas
\item mediante una tabla hash los switches saben a que segmento pertenecen las estaciones
\item cuando llega una trama al puente se fija la hora y actualiza el puerto si cambio. por si se modificaran las topologias
\item si no conoce por cual puerto enviar una trama. se envia a todos excepto por el que vino
\item \textbf{conmutacion al vuelo}: es posible empezar a reenviar ni bien se lea la cabezera de una trama, que contiene la direccion
\end{itemize}
\subsubsection{puentes con arbol de expansion}
\label{sec:org281eefc}
\begin{itemize}
\item enlaces redundantes. si se corta uno la red no se dividira en dos. pero crea ciclos en la topologia
\item los puentes ejecutan un algoritmo distribuido para construir el arbol
\item incluyen la distancia desde la raiz para recordar la ruta mas corta. desactivan los puertos que no formen parte de esa ruta
\end{itemize}
\subsubsection{redes lan virtuales}
\label{sec:org3898c44}
\begin{itemize}
\item agrupar a los usuarios en diferentes lan para reflejar la estructura de la organizacion
\item seguridad: por ejemplo separar servidores de computadoras de uso publico
\item carga: algunas lan se usan mucho mas que otras
\item trafico de difusion
\item las redes vlan se basan en switches diseñados para este proposito. el administrador decide cuantas vlan habra y como se llamaran
\item tablas de configuracion en los puentes. que vlan se puede acceder por un puerto
\item estandar 802.1q
\begin{itemize}
\item se cambio el encabezado de ethernet. tiene una nueva etiqueta vlan
\item los campos de vlan no los deben ver los usuarios, solo puntes y conmutadores
\item cuando una trama llega al primer switch con soporte para vlan agrega los campos y el ultimo los elimina
\end{itemize}
\end{itemize}
\section{capa de red}
\label{sec:orgbb29d07}
\subsection{aspecto de diseño}
\label{sec:org1bded8e}
\subsubsection{conmutacion de paquetes de almacenamiento y reenvio}
\label{sec:orgddf9eee}
\subsubsection{servicios proporcionados a la capa de transporte}
\label{sec:org4896b15}
\begin{itemize}
\item independientes de la tecnologia del enrutador
\item la capa de transporte debe estar aislada del tipo, cantidad y topologia de enrutadores
\item plan de numeracion uniforme para las direcciones disponibles
\end{itemize}
\subsubsection{implementacion del servicio sin conexion}
\label{sec:orga7589f1}
\begin{itemize}
\item los paquetes se transmiten por separado y se enrutan de manera independiente
\item datagramas
\item ip
\end{itemize}
\subsubsection{implementacion del servicio orientado a la conexion}
\label{sec:orge62565f}
\begin{itemize}
\item evitar la necesidad de elegir una nueva ruta para cada paquete enviado. cuando se establece una conexion se guarda la ruta
\item mpls: usa vez que se establece el circuito virtual los enrutadores intermedios asignan identificadores diferentes para origenes diferentes para diferenciarlos en una misma ruta
\end{itemize}
\subsection{algoritmos de enrutamiento}
\label{sec:orgb6b8040}
\begin{itemize}
\item un enrutador tiene dos procesos internos: uno maneja cada paquete conforme llega y busca en la tabla de ruteo la linea de salida. el otro es llenar y actualizar las tablas de ruteo, y ahi es donde entra el algoritmo de ruteo
\item muchas redes intentan reducir el numero de saltos que debe dar un paquete
\item no adaptativos: no basan sus decisiones en mediciones de trafico y topologia actuales. las rutas se elijen de antemano. enrutamiento estatico
\item adaptativos: no no adaptativos. enrutamiento dinamico
\end{itemize}
\subsubsection{principio de optimizacion}
\label{sec:orga9216d3}
\begin{itemize}
\item si una ruta es optima para i->j->k, tambien es optima para j->k
\item arbol sumidero: el conjunto de rutas optimas
\end{itemize}
\subsubsection{algoritmo de la ruta mas corta}
\label{sec:org0104644}
\begin{itemize}
\item ver la red como un grafo y buscar el camino mas corto
\item corto puede ser el numero de saltos, distancia geografica, u otras metricas
\end{itemize}
\subsubsection{inundacion}
\label{sec:orgb3fdcef}
\begin{itemize}
\item tecnica local. el enrutador envia por todas las lineas excepto por la que vino el paquete
\item gran cantidad de duplicados
\item numero maximo de saltos en la cabecera
\item numero de secuencia en paquetes para no enviarlos dos veces
\item no es practico para la mayoria de envios. pero tienen usos importantes como la difusion, porque asegura que todas las estaciones reciban el paquete
\item es en extremo robusta
\item requiere poca configuracion
\item siempre encuentra la ruta mas corta, sin contar el congestionamiento que provoca el algoritmo
\end{itemize}
\subsubsection{enrutamiento por vector de distancia}
\label{sec:org02d1b91}
\begin{itemize}
\item cada enrutador mantiene un vector (una tabla) con la mejor ruta para cada destino. esta tabla se va actualizando
\item cada T ms cada enrutador manda a sus vecinos su tabla
\item problema del conteo al infinito: la convergencia llega a la respuesta correcta, pero lo hace lentamente
\end{itemize}
\subsubsection{enrutamiento por estados de enlace}
\label{sec:org0eb4efd}
\begin{itemize}
\item las variantes is-is y ospf son usadas en la actualidad en internet
\item descrubrir a sus vecinos
\begin{itemize}
\item cuando un enrutador se pone en funcionamiento envia paquetes por todas las lineas que son respondidos con informacion de los vecinos
\end{itemize}
\end{itemize}
\subsection{ipv4}
\label{sec:org11416c0}
\begin{itemize}
\item conmutacion de paquetes
\item servicio sin conexion
\end{itemize}
\subsubsection{objetivos}
\label{sec:org6a6ff54}
\begin{itemize}
\item funcion de ruteo
\item transparencia en la red de redes
\item reglas de entrega de paquetes no confiable
\item unidad basica: datagrama
\end{itemize}
\subsubsection{clases de direcciones}
\label{sec:org393db5f}
\begin{itemize}
\item a: r.h.h.h. 1.0.0.0 a 126.0.0.0
\item b: r.r.h.h. 128.0.0.0 a 191.255.0.0
\item c: r.r.r.h. 192.0.0.0 a 223.255.255.0
\item d: multicast address. 224.0.0.0 a 239.255.255.255
\item e: reservado. 240.0.0.0 a 255.255.255.255
\item el primer octeto se da por el corrimiento del ultimo 1 de izquierda a derecha (0, 10, 110, 1110, 11110)
\end{itemize}
\subsubsection{packet switching}
\label{sec:org98326d3}
\begin{itemize}
\item el paquete se divide en el origen en unidades manejables: datagramas
\item los datagramas viajan al destino
\item se ensamblan en el destino para lograr el mensaje original
\item los paquetes se dividen segun los requisitos de cada punto intermedio (cada router)
\end{itemize}
\subsubsection{ruteo}
\label{sec:org9a1677c}
\begin{itemize}
\item proceso de seleccion del camino de un paquete
\item entrega directa: transmision entre hosts de una misma red ip. no necesita del router. se encapsula el datagrama en una trama y se envia directamente
\item entrega indirecta: los hosts se encuentran en redes separadas. se envia el datagrama a un ruteador de su red ip encapsulandolo en una trama
\item se compara el netid del transmisor con el de destino. si son iguales es entrega directa
\item sino usan las tablas de ruteo que indican por cada posible ip el siguiente salto que debe tomar en la ruta hasta el destino
\item las tablas tambien se usan para entrega directa
\end{itemize}
\subsubsection{direcciones privadas}
\label{sec:org324b07f}
\begin{itemize}
\item las ipv4 no alcanzan para todos los dispositivos del mundo
\item cada red interna usa un conjunto de ip privadas que se repiten en cada red que no sale a internet
\item por dentro la red se maneja con esas ip privadas, y desde afuera se ve una sola ip
\end{itemize}
\subsubsection{subredes ip}
\label{sec:org59503a3}
\begin{itemize}
\item cuando se usan bits de la parte de host para crear subredes
\end{itemize}
\subsection{icmp: internet control message protocol}
\label{sec:org0bc4e59}
\begin{itemize}
\item ip falla cuando el destino se desconecta de la red, cuando pasa el timeout para la respuesta o cuando router intermediarios estan muy congestionados
\item icmp es requerido por ip y debe ser incluido en una implementacion del protocolo
\item reporta errores, no corrige. aunque sugiere accioner a tomar
\end{itemize}
\subsubsection{funciones}
\label{sec:org41250b2}
\begin{itemize}
\item error: un nodo que reconoce un error de transmision genera un paquete icmp. este se reporta a la fuente original, que es la que esta en la cabecera del paquete. no puede avisar a los routers intermedios. ni el origen saber que router tuvo el problema
\item control: herramientas de diagnostico de la red (ping, traceroute)
\item trama \{ ip \{ icmp \{\} \} \}
\end{itemize}
\subsubsection{tipos}
\label{sec:orgefb6a97}
\begin{itemize}
\item 8/0 ping: solicitud eco/respuesta
\item 3 destination unreachable: cuando no puede entregar/direccionar un datagrama
\item 4 source quench: congestionamiento
\item 5 route change request: usado por el router directamente conectado host fuente para cambio de ruta
\item 11 time exceeded
\item 13/14: timestamp para sincronizacion, calculo de viaje redondo, etc
\item 17/18: solicitud/respuesta de mascara
\end{itemize}
\subsection{arp: address resolution protocol}
\label{sec:org24245e1}
\begin{itemize}
\item se usa para obtener direcciones mac, tanto para el ultimo paso (host destino) como para intermedios (routers)
\item el pedido es broadcast, la respuesta es unicast
\item el transmisor incluye su mac e ip para que los host actualicen
\item trama \{ arp \{\} \}
\item dos partes: transforma direcciones ip en direcciones fisicas. responde pedidos de otras maquinas
\item se mantiene una tabla con direcciones guardadas, que se actualizan cada cierto tiempo
\item por que se usa un broadcast que alcanza al destino para despues enviar un mensaje al mismo destino?: los mensajes broadcast son mas costosos porque cada maquina debe procesar el mensaje
\end{itemize}
\end{document}