% Created 2021-10-08 vie 22:21
% Intended LaTeX compiler: pdflatex
\documentclass[11pt]{article}
\usepackage[utf8]{inputenc}
\usepackage[T1]{fontenc}
\usepackage{graphicx}
\usepackage{grffile}
\usepackage{longtable}
\usepackage{wrapfig}
\usepackage{rotating}
\usepackage[normalem]{ulem}
\usepackage{amsmath}
\usepackage{textcomp}
\usepackage{amssymb}
\usepackage{capt-of}
\usepackage{hyperref}
\usepackage{fullpage}
\date{}
\title{práctica N3}
\hypersetup{
 pdfauthor={},
 pdftitle={práctica N3},
 pdfkeywords={},
 pdfsubject={},
 pdfcreator={Emacs 27.2 (Org mode 9.4.4)}, 
 pdflang={English}}
\begin{document}

\maketitle
\begin{enumerate}
\item \begin{itemize}
\item la cabecera de ipv6 tiene menos campos, es más simple
\item las direcciones de ipv6 son de 128 bits, las de ipv4 de 32
\item renombramiento de varios campos
\item tamaño de la cabecera de ipv6 es fijo (40 bytes), a diferencia del de ipv4
\end{itemize}
\item se usan para enviar opciones adicionales, que se insertan sólo si son necesarias

como la cabecera base es fija, se implementan las opciones fuera de ésta para optimizar la lectura de los paquetes

se indican por el campo de la cabecera base \emph{next header}, y el tamaño va en un campo en la cabecera de extensión
\item \begin{enumerate}
\item el host fuente envía los paquetes considerando su propio mtu de 1600

los paquetes llegan al primer router, y ve que la salida tiene un mtu menor, de 1400, entonces envía un icmp de error al comienzo para que se baje el mtu a 1400

de nuevo se envían los paquetes pero fragmentados a 1400

pasan el primer router, pero cuando llegan al segundo pasa lo mismo, el mtu es inferior, entonces se envía otra vez un icmp de error a la fuente pidiendo que reduzca el tamaño de fragmentación

el host fuente manda ahora con mtu 1300, y esta vez llega a destino porque la última salida del último router tiene un mtu mayor a 1300. los paquetes llegan a destino
\item la fragmentación se realiza en la máquina host, a diferencia de ipv4 que la realiza cada router
\end{enumerate}
\item \begin{enumerate}
\item FF01:0000:0000:0000:0000:0000:0000:0001

FF01::1
\item 2001:0000:1234:0000:0000:C1C0:ABCD:0876

2001:0:1234::C1C0:ABCD:0876

2001::1234:0:0:C1C0:ABCD:0876
\end{enumerate}
\item \begin{itemize}
\item unicast global
\begin{itemize}
\item globalmente ruteables. son las equivalentes a las ip públicas de ipv4
\item rango de 2000 a 3fff
\item formato:
\begin{center}
\begin{tabular}{rrrr}
3 & 45 & 16 & 64\\
001 & global routing prefix & subnet id & interface id\\
\end{tabular}
\end{center}
\end{itemize}
\item unicast link-local
\begin{itemize}
\item sólo válidas en el enlace local donde la interfaz está conectada
\item usadas para tareas administrativas, por ejemplo descubrimiento de vecinos
\item no se rutean
\item formato:
\begin{center}
\begin{tabular}{rrr}
16 & 48 & 64\\
fe80 & 0 & interface id\\
\end{tabular}
\end{center}
\end{itemize}
\item unicast unique-local
\begin{itemize}
\item ruteos internos dentro de un conjunto de enlaces
\item interconexión de redes sin conflictos, puede haber comunicación sin internet
\item no son ruteables en internet global
\item formato:
\begin{center}
\begin{tabular}{rrrrr}
7 & 1 & 40 & 16 & 64\\
fc00/7 & L & identificador global & subnet id & interface id\\
\end{tabular}
\end{center}
\item L=1: prefijo asignado local o L=0: prefijo asignado por la iana
\end{itemize}
\end{itemize}
\item \begin{enumerate}
\item 2001:db8:0:8:: a 2001:db8:0:f:ffff:ffff:ffff:ffff
\item 2001:0:0:ad00:: a 2001:0:0:adff:ffff:ffff:ffff:ffff
\end{enumerate}
\item 
\end{enumerate}
\end{document}