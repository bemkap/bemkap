% Created 2021-09-09 jue 19:23
% Intended LaTeX compiler: pdflatex
\documentclass[11pt]{article}
\usepackage[utf8]{inputenc}
\usepackage[T1]{fontenc}
\usepackage{graphicx}
\usepackage{grffile}
\usepackage{longtable}
\usepackage{wrapfig}
\usepackage{rotating}
\usepackage[normalem]{ulem}
\usepackage{amsmath}
\usepackage{textcomp}
\usepackage{amssymb}
\usepackage{capt-of}
\usepackage{hyperref}
\usepackage{fullpage}
\date{}
\title{práctica N1}
\hypersetup{
 pdfauthor={},
 pdftitle={práctica N1},
 pdfkeywords={},
 pdfsubject={},
 pdfcreator={Emacs 27.2 (Org mode 9.4.4)}, 
 pdflang={English}}
\begin{document}

\maketitle
\begin{enumerate}
\item a.
\begin{itemize}
\item punto a punto
\begin{itemize}
\item enlace permanente entre dos puntos finales
\end{itemize}
\item bus
\begin{itemize}
\item cada nodo está conectado a un cable central
\item todas las transmisiones en la red se realizan por este cable central o "bus"
\item más barata de implementar pero más dificil de manejar
\item si falla el bus la red queda dividida en dos
\end{itemize}
\item estrella
\begin{itemize}
\item cada nodo periférico se conecta a uno central llamado "switch" o "hub"
\item cliente-servidor
\item toda la comunicación pasa por el nodo central, que trabaja como repetidor
\item fácil de diseñar e implementar. simplicidad para agregar nodos
\item si falla el nodo central falla toda la red
\item \textbf{estrella extendida}: una estrella central y varias subredes con repetidores
\item \textbf{estrella distribuída}: varias subredes estrella conectadas cada una con la siguiente (daisy-chain)
\end{itemize}
\item anillo
\begin{itemize}
\item daisy-chain formando un bucle
\item los datos viajan sólo en una dirección
\item el rendimiento es mejor que el de la topología bus cuando hay mucha carga
\item no hay necesidad de un servidor
\item cuellos de botella
\item si un nodo no puede retransmitir la red falla
\end{itemize}
\item malla
\begin{itemize}
\item \textbf{totalmente conectada}
\item \textbf{parcialmente conectada}
\end{itemize}
\end{itemize}
b.
\begin{itemize}
\item bus: n, cada nodo con el cable central
\item estrella: n-1, cada nodo con el nodo central
\item anillo: n, cada uno con el siguiente y el último con el primero
\item malla: hasta \(\frac{n(n-1)}{2}}\), según sea totalmente o parcialmente conectada
\end{itemize}
\item a. si falla una conexión dos nodos no podrán comunicarse directamente, pero si a dos saltos. los demás se comunican sin problemas
b. un nodo no puede comunicarse con el central
c. si falla el bus la red se parte en dos
d. un fallo en cualquier conexión hace que no pueda haber comunicación
\item la frecuencia más alta es 20KHz, 20000 ciclos por segundo. por el teorema de nyquist el muestreo debe ser de por lo menos 40000 veces por segundo, el doble
\item \(C\approx 0.332*B*S/N=0.332*1000*24=7968\), casi 8kbit/s
\item a. bridge o switch. crea dos o más segmentos y si se quiere enviar de un segmento al mismo se corta la comunicación a los demás, alivianando un poco el tráfico
b. repetidor. sólo se quieren conectar dos dispositivos lejanos
c. hub. que repita la señal de una red a otra, ya que son pocos dispositivos
d. bridge o switch. como tienen diferentes estándares, hay que hacer alguna modificación a los datos de capa 1 para que sean entendidos por la otra red
\item 

\item la C está más cerca de A. el mensaje rts llega al ap B y a C pero no a D, que recién activa el nav cuando recibe el cts de B
\item a. subcapa control de acceso al medio
\begin{itemize}
\item iniciación: intercambio de tramas de control para establecer la disponibilidad de las estaciones. terminación: liberación de los recursos. identificación: saber dónde eviar o de dónde viene una trama
\item segmentación y agrupación: dividir o agrupar la información según la longitud de las tramas
\item sincronización octeto caracter: interpretar correctamente los bits. decodificarlos correctamente
\item delimitación de trama: separar las tramas para que sean entendidas por la otra terminal
\end{itemize}
b. subcapa control enlace lógico
\begin{itemize}
\item corrección de errores: implementar mecanismos para minimizar los errores que pueden surgir del ruido del medio
\item control de flujo: necesario para no saturar a un receptor con muchos emisores
\item recuperación de fallos: procedimientos para detectar situaciones inusuales como pérdida de tramas, tramas duplicadas o fuera de tiempo
\end{itemize}
\item DLE-STX-STX-DLE-DLE-ABC-DLE-ETX-DLE-BCD-DLE-STX

DLE-STX empieza la transmisión

DLE-DLE-ABC es DLE-ABC porque se usa un escape DLE

DLE-ETX termina la transmisión

se transmite STX-DLE-ABC
\end{enumerate}
\end{document}