\documentclass[12pt]{article}
\usepackage[utf8]{inputenc}
\usepackage{fullpage}
\title{{\textbf{\Large{Trabajo Práctico}\\Mini VM}}}
\author{Rossi, Martín. Legajo 9901/8}
\date{Febrero 2018}
\begin{document}
\maketitle
\section*{Problemas}
\begin{enumerate}
\item Encontrar una forma concisa de escribir los casos de cada instrucción, sean memoria, registro o inmediato.
\item Los registros son int y la memoria es char.
\item Al agregar call y ret el programa no necesariamente inicia en la primera instrucción del código.
\end{enumerate}
\section*{Soluciones}
\begin{enumerate}
\item Elegir con un array indexado por el tipo de operando.
\item La memoria se maneja manualmente como arreglo de ints.
\item Empezar por la etiqueta main. Cuando se procesan las etiquetas se guarda la ubicación de main y se empieza por ahí. Si no existe main empieza por el principio del código.
\end{enumerate}
\section*{Extensiones}
\begin{itemize}
\item Agregar secciones data y text, e instrucciones para reservar espacio para números o texto.
\item Agregar instrucciones más complejas de x86.
\item Usar el registro FLAGS para acarreo de operaciones o desbordamiento.
\item Agregar más registros, y simular la unidad de punto flotante.
\item Que la salida del programa sea un ejecutable para no hacer el parsing cada vez que se corre el programa.
\end{itemize}
\end{document}