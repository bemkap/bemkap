\documentclass[12pt]{article}
\usepackage{fullpage}
\usepackage{amsmath}
\usepackage{multirow}
\title{\LARGE \textbf{Desafío cinemática}}
\author{Franco Ferretti, Martín Rossi}
\date{}
\begin{document}
\maketitle
Partiendo de la aceleración $a(t)$, se obtienen las fórmulas de velocidad y desplazamiento:

$a(t)=a_0$

$v(t)=a_0*t+v_0$

$x(t)=\frac{a_0}{2}*t^2+v_0*t+x_0$\\

Los dos parten del reposo, entonces $v_0=0$.

Si se considera que los dos tienen un $x_0=0$, se puede igualar $x(t)$ a la altura y ver cuánto tarda cada uno en recorrer esa distancia.

Para el entusiasta $a_0=g$, entonces $H1=\frac{g}{2}*t^2\implies t=\sqrt{\frac{2*H1}{g}}$

Y superman inicia con cierta aceleración $a_0$, por lo tanto $H2=\frac{a_0}{2}*t^2\implies t=\sqrt{\frac{2*H2}{a_0}}$

Si superman tarda menos en recorrer $H2$ de lo que el entusiasta tarda en recorrer $H1$, llegará a salvarlo.

\begin{align*}
  \sqrt{\frac{2*H1}{g}}&>\sqrt{\frac{2*H2}{a_0}}\\
  \frac{2*H1}{g}&>\frac{2*H2}{a_0}\\
  a_0&>\frac{H2}{H1}g
\end{align*}

\end{document}