\documentclass[12pt,fleqn]{article}
\usepackage{fullpage}
\usepackage{amsmath}
\usepackage{multirow}
\usepackage{indentfirst}
\title{\LARGE \textbf{Trabajo práctico: unidad 5}\\\large Probabilidad y estadística}
\author{Martín Rossi}
\date{}
\begin{document}
\maketitle
\subsection*{1.}

$X: \textrm{tiempo de vida en años del marcapasos}$

$X \sim exp(\alpha)$

$E(X)=16$\\

Se sabe que si $X \sim exp(\alpha)$, entonces $E(X)=\frac{1}{\alpha}$, por lo tanto $E(X)=16=\frac{1}{\alpha}\implies \alpha=\frac{1}{16}$

$X \sim exp(\frac{1}{16})$\\

- ¿Cuál es la probabilidad de que a una persona a la que se le ha implantado este marcapasos se le deba reimplantar otro antes de 20 años?\\

$P(X<20)=F(20)=1-e^{-\frac{1}{16}*20} \approx 0.7135$

La probabilidad es de 71.35\% aproximadamente\\

- Si el marcapasos lleva funcionando correctamente 5 años en un paciente, ¿cuál es la probabilidad de que haya que cambiarlo antes de 25 años?\\
\begin{align*}
  P(X<25|X>5)&=\frac{P(X<25,X>5)}{P(X>5)}\tag{definición probabilidad condicional}\\
             &=\frac{F(25)-F(5)}{1-F(5)}\\
             &=\frac{1-e^{-\frac{1}{16}*25}-(1-e^{-\frac{1}{16}*5})}{1-(1-e^{-\frac{1}{16}*5})}\tag{$X \sim exp(\frac{1}{16})$}\\
             % &=\frac{-e^{-\frac{1}{16}*25}+e^{-\frac{1}{16}*5}}{e^{-\frac{1}{16}*5}}\\
             % &=-e^{-\frac{20}{16}}+1\\
             &\approx 0.7135
\end{align*}

La probabilidad es de 71.35\% aproximadamente
\newpage
\subsection*{2.}

$X: \textrm{duración en horas de un láser semiconductor a potencia constante}$\\

Se tiene el dato

$X \sim N(\mu=7000,\sigma=600)$\\

Si defino una nueva variable $Z=\frac{X-\mu}{\sigma}=\frac{X-7000}{600}$ entonces

$Z \sim N(0,1)$
\subsubsection*{a.} ¿Cuál es la probabilidad de que el láser falle antes de 5.000 horas?
\begin{align*}
  P_X(X<5000)&=P_X(\frac{X-7000}{600}<\frac{5000-7000}{600})\\
           &=P_Z(Z<-\frac{10}{3})\tag{Probabilidad de sucesos equivalentes}\\
           &=F_Z(-\frac{10}{3})\\
           &=0.0004\tag{Tabla de distribución normal estandarizada}
\end{align*}

La probabilidad de que falle antes de las 5000 horas es 0.04\%
\subsubsection*{b.} ¿Cuál es la duración en horas excedida por el 95\% de los láseres?\\

Lo que hay que buscar es $x$ tal que:

$P_X(X>x)=0.95 \iff P_X(X\le x)=0.05 \iff F_X(x)=0.05$\\

Para ello primero se busca el valor $z$ tal que $F_Z(z)=0.05$

En la tabla están los valores para 0.0495 y 0.0505, que son -1.65 y -1.64 respectivamente, por eso se tomará el valor $z=-1.645$ para el cual $F_Z(z)\approx 0.05$\\

Entonces usando la probabilidad de sucesos equivalentes con $Z=H(X)=\frac{X-7000}{600}$
\[0.05 \approx F_Z(-1.645)=F_X(-1.645*600+7000)=F_X(6013)\]

La duración excedida por el 95\% es de 6013 horas aproximadamente
\newpage
\subsubsection*{c.} Si se hace uso de tres láseres en un producto y se supone que fallan de manera independiente. ¿Cuál es la probabilidad de que tres sigan funcionando después de 7.000 horas?\\

Se calcula la probabilidad de que uno funcione más de 7000 horas
\begin{align*}
  P_X(X>7000)&=1-P_X(X\le 7000)\\
             &=1-0.5\tag{1}\\
             &=0.5
\end{align*}
(1) es 0.5 porque la distribución normal tiene media (y mediana) $\mu=7000$, y la mediana es el valor $m$ tal que P($X\le m$)=0.5\\

Como los sucesos son independientes la probabilidad de que los 3 sigan funcionando es el producto de las probabilidades de cada uno, o sea, $0.5^3=0.125$
\subsection*{3.}
$T_c: \textrm{temperatura en grados centígrados a la que está expuesta una computadora en el campo}$

$T_c \sim U[15,21]$
\[f_{T_c}(x)=\begin{cases}
    \frac{1}{21-15}=\frac{1}{6}&15\le x\le 21\\
    0&sino
  \end{cases}
\]

$T_f: \textrm{temperatura en grados fahrenheit a la que está expuesta una computadora en el campo}$

$T_f=\frac{9}{5}T_c+32$\\

$T_f$ es obtenida a partir de una función de variable aleatoria $H$ sobre $T_c$ ($T_f=H(T_c)$), donde $H(X)=\frac{9}{5}X+32\implies H^{-1}(X)=\frac{5}{9}(X-32)$
\begin{align*}
  F_{T_f}(x)&=P_{T_f}(T_f \le x)\\
            &=P_{T_c}(\frac{9}{5}T_c+32\le x)\tag{Probabilidad de sucesos equivalentes}\\
            &=P_{T_c}(T_c \le \frac{5}{9}(x-32))\\
            &=F_{T_c}(\frac{5}{9}(x-32))\\
            &=\begin{cases}
              0&x<59\\
              \frac{\frac{5}{9}(x-32)-15}{6}&59\le x\le\frac{349}{5}\tag{$15\le \frac{5}{9}(x-32)\le 21\implies 59\le x\le\frac{349}{5}$}\\
              1&\frac{349}{5}<x
              \end{cases}
\end{align*}
La función de densidad de probabilidad $f_{T_f}$ es la derivada de $F_{T_f}$:
\[f_{T_f}(x)=\begin{cases}
    \frac{5}{54}&59\le x\le\frac{349}{5}\\
    0&sino
  \end{cases}
\]
\end{document}