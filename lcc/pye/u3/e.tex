\documentclass[12pt]{article}
\usepackage{fullpage}
\usepackage{amsmath}
\title{práctica 3}
\author{martín rossi}
\date{}
\begin{document}
\maketitle
\textbf{5.}

\textbf{a.}$A \cup B \cup C$

\textbf{b.}$-(A \cup B \cup C)$

\textbf{c.}$A \cap B \cap C$

\textbf{d.}$(A \cap B) \cup (A \cap C) \cup (B \cap C)$

\textbf{e.}$(A \cap B) \cup (A \cap C) \cup (B \cap C) \cup (A \cap B \cap C)$

\textbf{6.}
$\#S=2^3=8$

\textbf{a.}$A=\{(V,V,M),(V,M,M),(M,V,M),(M,M,M)\}$

\textbf{b.}$B=\{(V,V,V),(V,V,M),(V,M,V),(V,M,M)\}$

\textbf{c.}$C=\{(M,V,M),(M,M,M),(V,V,V),(V,V,M),(V,M,V),(V,M,M)\}$

\textbf{7.}

\textbf{a.}$S=\{(a,b)|a\leftarrow\{1,2,3,4,5,6\},b\leftarrow\{1,2,3,4,5,6\}\}$

\textbf{b.}

$A=\{(2*a,b)|a\leftarrow\{1,2,3\},b\leftarrow\{1,2,3,4,5,6\}\}$

$B=\{(a,2*b+1)|a\leftarrow\{1,2,3,4,5,6\},b\leftarrow\{0,1,2\}\}$

$C=\{(2*a,2*b)|a\leftarrow\{1,2,3\},b\leftarrow\{1,2,3\}\}$

$D=\{(2*a+1,2*b+1)|a\leftarrow\{0,1,2\},b\leftarrow\{0,1,2\}\}$
  
\textbf{8.}

$2/3=P(A \cup B)=P(A)+P(B)-P(A \cap B)=1/4+1/2-P(A \cap B)=3/4-P(A \cap B)$

$P(A \cap B)=3/4-2/3>0$

no son mutualmente excluyentes porque la interseccion no es vacia

\textbf{9.}

$P(A \cup B)=3/4$

$P(-B)=2/3$

$P(A \cap B)=1/4$

$P(B)=1-P(-B)=1-2/3=1/3$

$P(A)=P(A \cup B)-P(B)+P(A \cap B)=3/4-1/3+1/4=2/3$

$P(-A \cap B)=P(B)-P(A \cap B)=1/3-1/4=1/12$

\textbf{10.}

puede pasar

$S=\{1,2,3,4,5,6\}$

$A=\{1,2,3\}$

$B=\{4,5,6\}$

$P(A \cap B)=0$

$P(A)+P(B)=1/2+1/2=1$

\textbf{11.}

\textbf{a.}$P(A)=4/8=1/2$

\textbf{b.}$P(B)=4/8=1/2$

\textbf{c.}$P(C)=6/8=3/4$

\textbf{d.}$P(B)=18/36=1/2$

\textbf{12.}

\textbf{a.}

hay 5 mayores a 31. 4 se pueden elegir de $\binom{5}{4}=5$ formas

hay 8 personas en total. 4 se pueden elegir de $\binom{8}{4}=70$ formas

$P(\textrm{"elegir todos de mas de 31"})=5/70=1/14$

\textbf{b.}

mismo razonamiento

$P(\textrm{"ningun arquitecto"})=\binom{6}{4}/\binom{8}{4}=15/70=3/14$

\textbf{13.}

\textbf{a.}

hay 4 mujeres. se elige el presidente entre 4 y despues los demas cualquiera

$4*7*6*5=840$

de un total de $8*7*6*5=1680$

$P(\textrm{``mujer presidente''})=1/2$

\textbf{b.}

mismo razonamiento. elijo primero al tesorero, que hay 2

$2*7*6*5=420$

$P(\textrm{"tesorero mayor a 50"})=420/1680=1/4$

\textbf{c.}

$2*1*5*4=40$

$P(..)=40/1680=1/42$

\textbf{14.}

$\textrm{total}=5*4*3*2*1=120$

\textbf{a.}

$P(\textrm{"hombres extremo"})=2*1*3*2*1/120=1/10$

\textbf{b.}

$P(\textrm{"alternado"})=3*2*2*1*1/120=1/10$

\textbf{c.}

$P(\textrm{"margarita centro"})=4*3*2*1/120=1/5$

\textbf{d.}

$P(\textrm{"margarita centro y manuel extremo"})=3*2*1/120=1/20$

\textbf{15.}

$\textrm{total}=120$

elijo las dos vocales primero, despues cualquiera: 2*1*3*2*1=12 combinaciones

eso si fuera primera y segunda letra nada mas, pero hay 4 pares contiguos
\begin{align*}
  P(\textrm{"dos vocales juntas"})&=(\textrm{combinaciones en un par}*\textrm{cantidad de pares})/120\\
                                  &=12*4/120=48/120=2/5
\end{align*}

\textbf{16.}

$\textrm{total}=2^4=16$

\textbf{a.}

$P(\textrm{"al menos una cara"})=1-P(\textrm{"ninguna cara"})=1-1/16=15/16$

\textbf{b.}

$P(\textrm{"a lo sumo tres cruces"})=1-P(\textrm{"cuatro cruces"})=1-1/16=15/16$

\textbf{c.}

$\binom{4}{2}/16=6/16=3/8$

\textbf{17.}

$\#S=6^5=7776$

$G: \textrm{``obtener generala''}$

$P: \textrm{``poker''}$

$G=\{(1,1,1,1,1),(2,2,2,2,2),(3,3,3,3,3),(4,4,4,4,4),(5,5,5,5,5),(6,6,6,6,6)\}$

$P(G)=\#G/\#S=1/1296$

contando la cantidad de poker posibles:

para poker de $1$ fijo los 4 $1s$, hay $\binom{5}{4}=5$ formas de fijarlos, y el otro dado puede ser cualquiera menos otro $1$, 5 posibilidades mas. es decir hay $5*5=25$ formas de hacer poker de 1. multiplicando por posibles numeros de dado queda $5*5*6=150$. $P(P)=150/7776$

\textbf{18.}

4 bolas. 3 blancas y 1 negra

si b0 a b2 son blancas y b3 negra

$S=\{(b0,b1),(b0,b2),(b0,b3),(b1,b2),(b1,b3),(b2,b3)\}$

$E=\{(b0,b1),(b0,b2),(b1,b2)\}$

$P(E)=\#E/\#S=1/2$

\textbf{19.}

\textbf{a.}

A.$(187+413)/1000=3/5$

B.$2/5$

C.$(187+113)/1000=3/10$

D.$7/10$

E.$113/1000$

\textbf{b.}

$113/(113+287)=113/400$

\textbf{20.}

$P(A)=0.2$

$P(B)=0.16$

$P(C)=0.14$

$P(A \cap B)=0.08$

$P(A \cap C)=0.05$

$P(B \cap C)=0.04$

$P(A \cap B \cap C)=0.02$

\textbf{a.}
\begin{align*}
  P(\overline{A \cup B \cup C})&=1-P(A \cup B \cup C)\\
             &=1-(P(A)+P(B)+P(C)-P(A \cap B)-P(A \cap C)-P(B \cap C)+P(A \cap B \cap C))\\
             &=1-(0.2+0.16+0.14-0.08-0.05-0.04+0.02)\\
             &=1-0.35\\
             &=0.65
\end{align*}

\textbf{b.}

$P(A \cup B \cup C)=0.35$

% c.
% d.
\textbf{21.}

para cada suma de 3 dados hay que contar tambien los 6 ordenes posibles, por ejemplo para 126 hay que contar tambien 162,612,621,216 y 261
ahora la cantidad tiros que suman 10 son 27 y los que suman 9 son 25. no tienen la misma probabilidad de ocurrir

\textbf{22.}

hay 3 reyes, porque 3/0.15=20 es el unico entero entre 1,2,3 y 4 reyes posibles. y 20 cartas en total

20*0.3=6. hay 6 bastos

20*0.6=12. hay 12 que no son bastos ni reyes

\textbf{a.}

hay 12 que no son basto ni rey. sobran 8

de las 8, 3 son reyes y 6 son bastos, lo que suman 9

osea que hay una que es rey y basto. el rey de basto esta entre las cartas

la probabilidad de extraerla es 1/20

\textbf{b.}

20

\textbf{25.}

A: ``falla A''

B: ``falla B''

$P(\overline{A} \cap B)=0.15$

$P(A)=0.2$

$P(A \cap B)=0.15$

\textbf{a.}
\begin{align*}
  P(A|B)&=\frac{P(A \cap B)}{P(B)}\\
        &=\frac{0.15}{P(B)}\\
        &=\frac{0.15}{0.3}\tag{$0.3=P(B \cap A)+P(B \cap \overline{A})=P(B)$}\\
        &=0.5
\end{align*}

\textbf{28.}

$P(A)=0.5$

$P(B)=0.3$

$P(C)=0.2$

$P(E|A)=0.8$

$P(E|B)=0.4$

$P(E|C)=0.1$

\textbf{a.}

$A,B,C$ particion entonces:
\begin{align*}
  P(E)&=P(E|A)P(A)+P(E|B)P(B)+P(E|C)P(C)\\
      &=0.8*0.5+0.4*0.3+0.1*0.2\\
      &=0.54
\end{align*}

\textbf{b.}
\begin{align*}
  P(A|E)&=\frac{P(E|A)P(A)}{P(E)}\\
        &=\frac{0.8*0.5}{0.54}\\
        &=0.7407
\end{align*}
\end{document}