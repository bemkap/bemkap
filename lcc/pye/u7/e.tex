\documentclass[12pt,fleqn]{article}
\usepackage{fullpage}
\usepackage{amsmath}
\usepackage{amssymb}
\usepackage{indentfirst}
\title{práctica 7}
\author{martín rossi}
\date{}
\begin{document}
\maketitle
\subsection*{1.}
\subsubsection*{a.}
D: producto defectuoso

N: producto no defectuoso

$P(D)=p, P(N)=1-p$

proceso estocástico $\{N_n:n\in\mathbb{N}\}$

$\Omega=\{w:w=(w_1,w_2,...), w_i\in\{D,N\}\}$

$N_n(w)\sim Bin(n,p)$

$P(N_n=k)=\binom{n}{k}p^k(1-p)^{n-k}$
\subsubsection*{b.}
-Cantidad de estados discreto, la cantidad de productos defectuosos($\mathbb{N}_0$)

-Instantes de observación discretos, cada vez que se fabrica un producto($\mathbb{N}_0$)

-Propiedad markoviana:
\subsubsection*{c.}
$E=\mathbb{N}_0$\\

$P=\begin{pmatrix}
  1-p&p&0&0&0&0&...\\
  0&1-p&p&0&0&0&...\\
  0&0&1-p&p&0&0&...\\
  0&0&0&1-p&p&0&...\\
  \vdots&\vdots&\vdots&\vdots&\vdots&\vdots&\\
\end{pmatrix}$
\subsubsection*{d.}
Los estados son todos transitorios, porque para cada estado hay una probabilidad no nula de abandonarlo, y no regresar nunca porque cada vez se van agregando más productos defectuosos.
\subsubsection*{e.}
$N_n \sim Bin(n,p)$ entonces:

$E(N_n)=np$

$V(N_n)=np(1-p)$
\subsection*{2.}
\subsubsection*{a.}
\begin{align*}
  P(N_5=1,N_4=0,N_3=0,N_2=0,N_1=0)&=p(0,0)p(0,0)p(0,0)p(0,0)p(0,1)p(N_0=0)\\
                                        &=0.38^40.62
\end{align*}
\subsubsection*{b.}
\begin{align*}
  P(N_5=2|N_4=1)P(N_4=1)=0.62*0.136083=0.0843712
\end{align*}
\subsubsection*{c.}
$P(N_{20}=12|N_{12}=9)=P(N_8=3)=P^8(0,3)=0.0602776$
\subsubsection*{d.}
38 automóviles
\subsection*{4.}
$E=\{S,N\}$\\

$P=\begin{pmatrix}
  0.9&0.1\\
  0.2&0.8
\end{pmatrix}$
\subsection*{5.}
\subsubsection*{a.}
\begin{align*}
P\{X_1 = b, X_2 = b, X_3 = b, X_4 = a, X_5 = c|X_0 = a\}&=P(a,b)P(b,b)P(b,b)P(b,a)P(a,c)\\
                                                        &=\frac{1}{3}\frac{3}{4}\frac{3}{4}\frac{1}{4}\frac{2}{3}\\
                                                        &=\frac{1}{32}
\end{align*}
\subsubsection*{b.}
\begin{align*}
  P\{X_1 = b, X_3 = a, X_4 = c, X_6 = b|X_0 = a\}&=P(a,b)P^2(b,a)P(a,c)P^2(c,b)\\
                                                 &=\frac{1}{3}0.1875\frac{2}{3}\frac{4}{3}\\
                                                 &\approx 0.05
\end{align*}
\subsubsection*{c.}
\begin{align*}
  P \{X_2 = b, X_5 = b, X_6 = a\}&=P \{X_2 = b, X_5 = b, X_6 = a|X_0=a\}P(X_0=a)\\
                                 &+P \{X_2 = b, X_5 = b, X_6 = a|X_0=b\}P(X_0=b)\\
                                 &+P \{X_2 = b, X_5 = b, X_6 = a|X_0=c\}P(X_0=c)\\
                                 &=P^2(a,b)P^3(b,b)P(b,a)P(X_0=a)\\
                                 &+P^2(b,b)P^3(b,b)P(b,a)P(X_0=b)\\
                                 &+P^2(c,b)P^3(b,b)P(b,a)P(X_0=c)\\
                                 &=0.25*0.5468*0.25*0.4\\
                                 &+0.6458*0.5468*0.25*0.2\\
                                 &+0.1333*0.5468*0.25*0.4\\
\end{align*}
\subsection*{6.}
\subsubsection*{a.}
$E=\{0,1,2\}$

$P=\begin{pmatrix}
  0&0.5&0.5\\
  0.75&0&0.25\\
  0&0&1
\end{pmatrix}$
\subsubsection*{c.}
$(\pi_0P^\infty)=\{0.43,0.27,0.3\}$
\subsection*{8.}
\subsubsection*{b.}
$E=\{a,b,c,d,e,f\}$

\begin{tabular}{|r|r|r|r|r|r|r|}
  \hline
  F(i,j)&a&b&c&d&e&f\\\hline
  a&1&1&0&0&0&0\\\hline
  b&1&1&0&0&0&0\\\hline
  c&0&0&1&1&0&0\\\hline
  d&0&0&1&1&0&0\\\hline
  e&1&1&0&0&0&0\\\hline
  f&1&1&0&0&$\frac{1}{3}$&0\\\hline
\end{tabular}
\subsubsection*{c.}
\begin{tabular}{|r|r|r|r|r|r|r|}
  \hline
  R(i,j)&a&b&c&d&e&f\\\hline
  a&$\infty$&$\infty$&0&0&&\\\hline
  b&$\infty$&$\infty$&0&0&&\\\hline
  c&0&0&$\infty$&$\infty$&0&0\\\hline
  d&0&0&$\infty$&$\infty$&0&0\\\hline
  e&&&0&0&&\\\hline
  f&&&0&0&&\\\hline
\end{tabular}
  
\end{document}