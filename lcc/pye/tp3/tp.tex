\documentclass[12pt]{article}
\usepackage{fullpage}
\usepackage{amsmath}
\usepackage{multirow}
\title{\LARGE \textbf{Trabajo práctico: unidad 3}\\\large Probabilidad y estadística}
\author{Martín Rossi}
\date{}
\begin{document}
\maketitle
\textbf{1)}
Se definen los sucesos:

$T_0:\textrm{``se emite un 0''}$

$T_1:\textrm{``se emite un 1''}$

$R_0:\textrm{``se recibe un 0''}$

$R_1:\textrm{``se recibe un 1''}$

Se tiene que:

$P(T_0)=0.5$

$P(T_1)=0.5$

Entonces se define el suceso $E:\textrm{``se comete un error en la transmisión"}$ como:

$E=(T_0 \cap R_1) \cup (T_1 \cap R_0)$

Y se calcula la probabilidad:
\begin{align*}
  P(E)&=P((T_0 \cap R_1) \cup (T_1 \cap R_0))\\
      &=P(T_0 \cap R_1)+P(T_1 \cap R_0)-P(T_0 \cap R_1 \cap T_1 \cap R_0)\\
      &=P(T_0 \cap R_1)+P(T_1 \cap R_0)\tag{$T_0$ y $T_1$ excluyentes}\\
      &=P(T_0|R_1)P(R_1)+P(T_1|R_0)P(R_0)\\
      &=P(T_0)P(R_1)+P(T_1)P(R_0)\tag{si se asume independencia entre $R$ y $T$}\\
      &=0.5*P(R_1)+0.5*P(R_0)
\end{align*}

\textbf{2)}
Se definen los sucesos:

$I_1: \textrm{"error importante en la primer prueba"}$

$I_2: \textrm{"error importante en la segunda prueba"}$

$M_1: \textrm{"error menor en la primer prueba"}$

$M_2: \textrm{"error menor en la segunda prueba"}$

$N_1: \textrm{"ningún error en la primer prueba"}$

$N_2: \textrm{"ningún error en la segunda prueba"}$

Se tiene que:

$P(I_1)=0.6$

$P(M_1)=0.3$

$P(N_1)=0.1$
\newpage
\textbf{a)}
Usando estas probabilidades junto con la fórmula de probabilidad condicional

$P(A|B)=P(A \cap B)/P(B)$ se forma la tabla de intersecciones:\\

\begin{tabular}{r r r r r}
  &&\multicolumn{3}{l}{\textbf{Tipo de error segunda prueba}}\\
  &&Importante&Menor&Ninguno\\
  \textbf{Tipo de error}&Importante&0.18&0.3&0.12\\
  \textbf{primera prueba}&Menor&0.03&0.09&0.18\\
  &Ninguno&0&0.02&0.08\\
\end{tabular}\\

\textbf{b)}
Se puede condicionar la probabilidad por el resultado de la primer prueba.

Como $I_1,M_1,N_1$ forman una partición de $S$, se calcula $P(I_2)$ con los valores de la tabla:
\begin{align*}
  P(I_2)&=P(I_2|I_1)P(I_1)+P(I_2|M_1)P(M_1)+P(I_2|N_1)P(N_1)\\
        &=0.3*0.6+0.1*0.3+0*0.1\\
        &=0.21
\end{align*}

\textbf{c)}
\begin{align*}
  P(M_1|I_2)&=P(M_1 \cap I_2)/P(I_2)\\
            &=0.03/0.21\\
            &=0.1429
\end{align*}

\textbf{d)}
\begin{align*}
  P(M_2)&=P(M_2|I_1)P(I_1)+P(M_2|M_1)P(M_1)+P(M_2|N_1)P(N_1)\\
        &=0.5*0.6+0.3*0.3+0.2*0.1\\
        &=0.41
\end{align*}
\begin{align*}
  P(N_2)&=P(N_2|I_1)P(I_1)+P(N_2|M_1)P(M_1)+P(N_2|N_1)P(N_1)\\
        &=0.2*0.6+0.6*0.3+0.8*0.1\\
        &=0.38
\end{align*}

\begin{tabular}{r r r r r}
  &&\multicolumn{3}{l}{\textbf{Tipo de error primera prueba}}\\
  &&Importante&Menor&Ninguno\\
  \textbf{Tipo de error}&Importante&0.8571&0.1429&0\\
  \textbf{segunda prueba}&Menor&0.7317&0.2195&0.0488\\
  &Ninguno&0.3158&0.4737&0.2105
\end{tabular}\\

Ningún resultado de la primera es independiente al de la segunda, una vez hechas

las dos pruebas.
\newpage
\textbf{3)}
Se definen los sucesos:

$E: \textrm{``una persona tiene la enfermedad''}$

$A: \textrm{``el test da positivo''}$

Tenemos los siguientes datos:

$P(A|E)=0.9$

$P(A|\overline{E})=0.05$

$P(E)=0.12$

$P(\overline{E})=0.88$

Se calcula $P(\overline{E}|A)$:

\begin{align*}
  P(\overline{E}|A)&=\frac{P(A|\overline{E})P(\overline{E})}{P(A)}\tag{Teorema de Bayes}\\
                   &=\frac{0.05*0.88}{P(A|E)P(E)+P(A|\overline{E})P(\overline{E})}\\
                   &=\frac{0.05*0.88}{0.9*0.12+0.05*0.88}\\
                   &=0.2895
\end{align*}
\end{document}