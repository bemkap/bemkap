% Created 2021-09-25 sáb 13:11
% Intended LaTeX compiler: pdflatex
\documentclass[11pt]{article}
\usepackage[utf8]{inputenc}
\usepackage[T1]{fontenc}
\usepackage{graphicx}
\usepackage{grffile}
\usepackage{longtable}
\usepackage{wrapfig}
\usepackage{rotating}
\usepackage[normalem]{ulem}
\usepackage{amsmath}
\usepackage{textcomp}
\usepackage{amssymb}
\usepackage{capt-of}
\usepackage{hyperref}
\usepackage{fullpage}
\date{}
\title{práctica N1}
\hypersetup{
 pdfauthor={},
 pdftitle={práctica N1},
 pdfkeywords={},
 pdfsubject={},
 pdfcreator={Emacs 27.2 (Org mode 9.4.4)}, 
 pdflang={English}}
\begin{document}

\maketitle
\begin{enumerate}
\item \begin{enumerate}
\item \begin{itemize}
\item punto a punto
\begin{itemize}
\item enlace permanente entre dos puntos finales
\end{itemize}
\item bus
\begin{itemize}
\item cada nodo está conectado a un cable central
\item todas las transmisiones en la red se realizan por este cable central o "bus"
\item más barata de implementar pero más dificil de manejar
\item si falla el bus la red queda dividida en dos
\item cable coaxil
\item para agregar nodos hay que conectarse al bus central
\end{itemize}
\item estrella
\begin{itemize}
\item cada nodo periférico se conecta a uno central llamado "switch" o "hub"
\item cliente-servidor
\item toda la comunicación pasa por el nodo central, que trabaja como repetidor
\item fácil de diseñar e implementar. simplicidad para agregar nodos
\item si falla el nodo central falla toda la red
\item \textbf{estrella extendida}: una estrella central y varias subredes con repetidores
\item \textbf{estrella distribuída}: varias subredes estrella conectadas cada una con la siguiente (daisy-chain)
\item mucho flujo de datos
\item par trenzado
\end{itemize}
\item anillo
\begin{itemize}
\item daisy-chain formando un bucle
\item los datos viajan sólo en una dirección
\item el rendimiento es mejor que el de la topología bus cuando hay mucha carga
\item no hay necesidad de un servidor
\item cuellos de botella
\item si un nodo no puede retransmitir la red falla si es half duplex
\item fibra óptica
\item no es sencillo agregar nuevos nodos
\end{itemize}
\item malla
\begin{itemize}
\item \textbf{totalmente conectada}
\item \textbf{parcialmente conectada}
\item mucho flujo de datos y redundancia
\end{itemize}
\end{itemize}
\item \begin{itemize}
\item bus: 1, cada nodo con el cable central
\item estrella: n-1, cada nodo con el nodo central
\item anillo: n, cada uno con el siguiente y el último con el primero
\item malla: hasta \(\frac{n(n-1)}{2}\), según sea totalmente o parcialmente conectada
\end{itemize}
\end{enumerate}
\item \begin{enumerate}
\item si falla una conexión dos nodos no podrán comunicarse directamente, pero si a dos saltos. los demás se comunican sin problemas
\item un nodo no puede comunicarse con el central
\item si falla el bus la red se parte en dos
\item un fallo en cualquier conexión hace que no pueda haber comunicación
\end{enumerate}
\item la frecuencia más alta es 20KHz, 20000 ciclos por segundo. por el teorema de nyquist el muestreo debe ser de por lo menos 40000 veces por segundo, el doble
\item \(C\approx 0.332*B*S/N=0.332*1000*24=7968\), casi 8kbit/s
\item \begin{enumerate}
\item bridge o switch. crea dos o más segmentos y si se quiere enviar de un segmento al mismo se corta la comunicación a los demás, alivianando un poco el tráfico
\item repetidor. sólo se quieren conectar dos dispositivos lejanos
\item hub. que repita la señal de una red a otra, ya que son pocos dispositivos
\item bridge o switch. como tienen diferentes estándares, hay que hacer alguna modificación a los datos de capa 1 para que sean entendidos por la otra red
\end{enumerate}
\item \begin{enumerate}
\item distancia: algunos medios tienen más alcance que otros
\item ambiente: condiciones ambientales determinan que medio es más conveniente usar, por ejemplo si es interior o exterior, clima, agua o tierra
\end{enumerate}
\item la C está más cerca de A. el mensaje rts llega al B y a C pero no a D, que recién activa el nav cuando recibe el cts de B. D es una estación oculta para A
\item \begin{enumerate}
\item subcapa control de acceso al medio
\begin{itemize}
\item iniciación: intercambio de tramas de control para establecer la disponibilidad de las estaciones. terminación: liberación de los recursos. identificación: saber dónde eviar o de dónde viene una trama
\item segmentación y agrupación: dividir o agrupar la información según la longitud de las tramas
\item sincronización octeto caracter: interpretar correctamente los bits. decodificarlos correctamente
\item delimitación de trama: separar las tramas para que sean entendidas por la otra terminal
\end{itemize}
\item subcapa control enlace lógico
\begin{itemize}
\item corrección de errores: implementar mecanismos para minimizar los errores que pueden surgir del ruido del medio
\item control de flujo: necesario para no saturar a un receptor con muchos emisores
\item recuperación de fallos: procedimientos para detectar situaciones inusuales como pérdida de tramas, tramas duplicadas o fuera de tiempo
\end{itemize}
\end{enumerate}
\item DLE-STX-STX-DLE-DLE-ABC-DLE-ETX-DLE-BCD-DLE-STX

DLE-STX empieza la transmisión

DLE-DLE-ABC es DLE-ABC porque se usa un escape DLE

DLE-ETX termina la transmisión

se transmite STX-DLE-ABC
\item \begin{enumerate}
\item sin conexión y sin acuse (tipo 1)
\item sin conexión y con acuse (tipo 3)
\item sin conexión y sin acuse (tipo 1)
\item con conexión (tipo 2)
\end{enumerate}
\item .
\begin{center}
\begin{tabular}{lll}
capa física(802.11) & --> & capa física(802.3)\\
subcapa mac(802.11) &  & subcapa mac(802.3)\\
subcapa llc &  & subcapa llc\\
\textbf{notebook} &  & \textbf{pc}\\
\end{tabular}
\end{center}
\item .
\begin{center}
\begin{tabular}{lllll}
 & fastbit & gigabit & fibra & lan inalambrica\\
velocidad & 100mbps & 1gbps & infinita & 100mbps\\
medio tx & par trenzado & fibra & fibra & aire\\
estandar & 802.3 & 802.3 & 802.3 & 802.11\\
\end{tabular}
\end{center}
\item el problema sucede cuando en una red la señal de una terminal A no llega a otra terminal B, por lo que los paquetes que envíe A no llegarán a B. se dice que B está oculta para A
\item \begin{enumerate}
\item no sería conveniente porque al haber estaciones ocultas no se podría determinar cuando hay colisiones
\end{enumerate}
\item \begin{enumerate}
\item a y b forman una red, c y d otra
\item tramas enviadas por un ap
\item a envía la última trama a b, c quiere empezar a enviar y d prende el nav
\end{enumerate}
\item no es posible que un mismo dispositvo sea maestro en dos piconets al mismo tiempo. el código de acceso se deriva de la identidad del maestro, por lo tanto no habría dos piconets, sino que sería una grande
\item la version 4 disminuyó la latencia de 100ms a 6ms, cambió la topología de piconet a estrella-bus y aumentó la cantidad de esclavos activos de 7 a ilimitado
\item en la versión clásica hay hasta 7 esclavos activos y la comunicación es síncronos. en ble es al revés
\item si soporta streaming de audio, a diferencia de los anteriores
\item 

\item \begin{itemize}
\item incrementar tasa de datos
\item reducir retardos
\item ancho de banda escalable
\item eficiencia espectral
\item arquitectura simplificada y con ip
\item diferentes tipos de usuarios
\item mejor consumo de energía
\item velocidad de bajada: 326Mbps con 4x4 antenas, 172Mbs con 2x2
\item velocidad de subida: 86Mbps
\end{itemize}
\item .
\begin{center}
\begin{tabular}{lll}
red troncal epc & ---> & red troncal epc\\
red acceso e-utran &  & red acceso e-utran\\
smartphone lte &  & smartphone lte\\
\end{tabular}
\end{center}
\item sí. lte está enteramente basado en el protocolo ip, tanto llamadas de voz como transmisión de datos
\end{enumerate}
\end{document}